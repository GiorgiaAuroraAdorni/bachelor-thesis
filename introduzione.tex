\chapter{Introduzione}
\label{chap:introduzione}
\Blindtext

Data mining: estrazione di significato da grandi quantità di dati
analisi dell'espressione umana che rendo conto da una parte del linguaggio come veicolo di definizione e di affermazione del se
evoluzione della comunicazione verso le nuove forme di dialogo tra persone e comunità

la raccolta di dadi e la necessità di produrre analisi sempre più velocemente ha fatto crescere in questa direzione la ricerca nel campo dell'apprendimento automatico o \emph{machine learning}.
In particolare, nell'ambito del text mining, si sono sviluppate metodologie che consentono ai computer di confrontarsi con il linguaggio umano, di elaborarlo e comprenderlo.


L'obiettivo di questa tesi è descrivere e sfruttare alcune tecniche come il machine learning e Natural Language Processing nell'ambito del ... 


Extroversion was the strongest predictor of leadership emergence — who becomes a leader — and leadership effectiveness — who's successful in a leadership position. But it was a better predictor of emergence than effectiveness.

What's more, when the study authors deconstructed extroversion into distinct parts, they found that dominance and sociability better predicted leadership than extroversion as a whole. This makes sense, the study authors write, "as both sociable and dominant people are more likely to assert themselves in group situations."
Conscientiousness, or a person's tendency to be organized and hard-working, was the second strongest predictor of leadership.
Again, conscientiousness was more closely linked to leader emergence than to leadership effectiveness. The authors write: "[T]he organizing activities of conscientious individuals (e.g. note taking, facilitating processes) may allow such individuals to quickly emerge as leaders.

In business settings, openness to experience is an important predictor of leadership. Justin Sullivan / Getty
Openness to experience was the third strongest predictor of leadership. However, it's worth noting that, in business settings specifically, openness was just as strongly linked to leadership as extroversion.
Neuroticism was not a strong predictor of leadership, meaning that highly neurotic people are not especially likely or unlikely to become leaders.
Agreeableness, or friendliness, was the "least relevant" to leadership of all the traits studied. Interestingly, however, when the researchers looked only at leadership effectiveness, agreeableness was related.




Psicologia e informatica:
I modelli computazionali possono essere utilizzati per modellare i sistemi come una scatola nera, ma possono anche essere usati per informare i modelli di elaborazione delle informazioni che mirano a comprendere la cognizione umana.




Le regioni cerebrali che codificano per vari tratti di personalità sono spesso accoppiate con regioni responsabili della comunicazione verbale e scritta. Inoltre, l'avvento dei social media e una comunità online sempre più connessa rendono sempre più disponibili i dati testuali personalizzati. In questo studio, ipotizziamo che uno stile di scrittura individuale è in gran parte accoppiato con i tratti della sua personalità e presenta un modello di apprendimento profondo per predire il tipo di personalità di Myers Briggs attraverso i dati testuali dei libri. Sviluppare un modello accurato e aprire questa domanda di ricerca avrebbe implicazioni significative nella business intelligence, nell'analisi della compatibilità delle relazioni e in altri campi della sociologia.










\chapter*{Abstract}
\label{Abstract}


Apprendimento della personalità basato sul linguaggio naturale. 
Reti neurali per la previsione dei tipi di personalità di Myers Brigg dagli stili di scrittura.






La personalità è l'essenza che definisce un individuo in quanto guida il modo in cui pensiamo, agiamo e interpretiamo stimoli esterni. 
Nel corso del secolo scorso, gli aspetti della personalità sono stati studiati da molti punti di vista, sia attraverso l'analisi delle relazioni interpersonali, delle dinamiche di gruppo e dei social network, sia attraverso le opere nelle neuroscienze che rivelano le basi biologiche dei tratti della personalità.


Il testo e la scrittura sono diventati un mezzo di comunicazione importante nell'era digitale. L'uso crescente dei social media e il fenomeno delle recensioni online hanno favorito un rapido aumento dei dati testuali digitali su cui vengono effettuate svariati tipi di analisi.
In particolare la convergenza tra scienze sociali (psicologiche) e informatiche hanno portato i ricercatori a sviluppare metodi automatizzati (approcci automatici) per estrarre e studiare le informazioni digitali contenute nel materiale testuale per prevedere i tratti della personalità.
La maggior parte degli attuali studi automatici di rilevamento della personalità si sono concentrati sul modello di personalità Big 5 come quadro per studiare le caratteristiche intrinseche dell'essere umano. 


Partendo da un dizionario di aggettivi che la letteratura psicologica definisce come marker dei cinque grandi tratti di personalità (Big Five), si vuole identificare un adeguato spazio semantico che permetta di definire la personalità dell’oggetto target a cui un determinato testo si riferisce. I dati che verranno utilizzati per definire lo spazio semantico e testare la sua funzionalità sono messi a disposizione da Yelp Dataset Challenge, che contiene 5˙200˙000 reviews relative a 174˙000 businesses di 11 aree metropolitane nel mondo. 




Obiettivo è la progettazione e lo sviluppo di un modello computazionale per l'apprendimento della personalità a partire dal linguaggio naturale.  


In questo studio, ipotizziamo che uno stile di scrittura individuale è in gran parte accoppiato con i tratti della sua personalità e presenta un modello di apprendimento profondo per predire il tipo di personalità di Myers Briggs attraverso i dati testuali dei libri. Sviluppare un modello accurato e aprire questa domanda di ricerca avrebbe implicazioni significative nella business intelligence, nell'analisi della compatibilità delle relazioni e in altri campi della sociologia.



La personalità è considerata uno degli argomenti di ricerca più influenti in psicologia perché è predittiva di molti esiti consequenziali come la salute mentale e fisica, la qualità delle relazioni interpersonali, l'adeguamento alla carriera e la soddisfazione, le prestazioni sul posto di lavoro e il benessere generale.

È ampiamente noto che i tratti della personalità come l'extraversione, la coscienziosità e il nevroticismo sono relativamente coerenti per tutta la vita. Tuttavia, i modi in cui i nostri comportamenti sono espressi attraverso le parole e l'azione non sono sempre determinati dai tratti soggettivi della personalità e dagli impulsi da soli.

Molte decisioni importanti, le dinamiche sociali e le decisioni politiche si basano anche sulla valutazione della personalità di un individuo con il quale non si è interagito molto personalmente. 

Ma leggere solo i comportamenti di altre persone non è sufficiente per fare previsioni accurate della loro personalità. Il compito diventa ancora più difficile quando si tenta di formulare giudizi basati solo sulla comunicazione scritta. 

Poiché il mondo si basa molto più sulla comunicazione basata sul testo rispetto alle interazioni faccia a faccia, sta diventando sempre più importante sviluppare modelli che possano leggere automaticamente e con precisione l'essenza di altri individui basandosi esclusivamente sulla scrittura. 

Fortunatamente, studi in neuroscienza hanno rivelato una mappatura vicina delle regioni del cervello responsabili dei tratti della personalità come l'extraversione e il nevroticismo, nonché quelli che sono legati alla comunicazione scritta.

I precedenti modelli di previsione della personalità si sono concentrati sull'applicazione di tecniche generali di apprendimento automatico e reti neurali per predire i tratti di personalità del Big Five di openness, conscientiousness, extraversion, agreeableness e neuroticism dai post sui social media. 

Gli studi che si focalizzano sui tratti del Big Five tendono a dare un tratto alla figura di un individuo.


In questo lavoro, abbiamo esplorato una varietà di metodi per affrontare il problema della predizione della personalità. 

Abbiamo iniziato costruendo manualmente un vasto corpus di brani tratti da romanzi famosi con autori di tipi MBTI. 

Per valutare la difficoltà di identificare gli MBTI dal testo, abbiamo raggruppato segmenti di testo basati su somiglianze di incorporamento di parole per determinare se esistesse una distribuzione non uniforme di tipi di personalità. Ciò fornisce una buona cornice iniziale di riferimento per capire quanto siano sottili i tratti della personalità quando sono nascosti nei dati scritti. 

Abbiamo poi implementato un sacco di reti neurali feed-forward come base per capire come i modelli semplici in deep-learning possano fornire informazioni sulle caratteristiche della personalità nascoste. 

Infine, ci addentriamo in una rete neuronale ricorrente basata sulla memoria a lungo termine più complessa e miriamo a costruire un sistema più generalizzabile che possa incorporare il significato della scrittura per determinare i tipi di personalità generali.



tratti di personalità Big 5.

I risultati delle analisi mostrano che il potere predittivo delle impronte digitali sui tratti della personalità è in linea con il 
Nel complesso, i nostri risultati indicano che la precisione delle previsioni è coerente tra i tratti Big 5 e che l'accuratezza migliora quando le analisi includono dati demografici e diversi tipi di impronte digitali.



struttura di personalità per guidare la nostra comprensione e rivelare il ruolo delle parole nel descrivere le caratteristiche di un utente. 
Questo studio preliminare ha rivelato come le percezioni del pubblico riguardo a parole specifiche possano aiutarci a rilevare la personalità. 

Come prima fase del nostro studio, questo esperimento si concentra sulla raccolta delle percezioni generali dei malesi verso 52 aggettivi inglesi categorizzazione delle parole sotto tratti PEN. 
La valutazione fornisce l'analisi necessaria che potrebbe aiutare la nostra ricerca principale che si concentra su rilevazioni automatiche della personalità. 










Abbiamo iniziato costruendo manualmente un vasto corpus di brani tratti da romanzi famosi con autori di tipi MBTI. 

Per valutare la difficoltà di identificare gli MBTI dal testo, abbiamo raggruppato segmenti di testo basati su somiglianze di incorporamento di parole per determinare se esistesse una distribuzione non uniforme di tipi di personalità. Ciò fornisce una buona cornice iniziale di riferimento per capire quanto siano sottili i tratti della personalità quando sono nascosti nei dati scritti. 



I risultati delle analisi mostrano che il potere predittivo delle impronte digitali sui tratti della personalità è in linea con il 
Nel complesso, i nostri risultati indicano che la precisione delle previsioni è coerente tra i tratti Big 5 e che l'accuratezza migliora quando le analisi includono dati demografici e diversi tipi di impronte digitali.


struttura di personalità per guidare la nostra comprensione e rivelare il ruolo delle parole nel descrivere le caratteristiche di un utente. 
Questo studio preliminare ha rivelato come le percezioni del pubblico riguardo a parole specifiche possano aiutarci a rilevare la personalità. 

Come prima fase del nostro studio, questo esperimento si concentra sulla raccolta delle percezioni generali dei malesi verso 52 aggettivi inglesi categorizzazione delle parole sotto tratti PEN. 
La valutazione fornisce l'analisi necessaria che potrebbe aiutare la nostra ricerca principale che si concentra su rilevazioni automatiche della personalità. 




