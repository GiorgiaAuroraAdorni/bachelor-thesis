\chapter{Contesto}
\label{chap:contesto}

Con il termine \emph{personalità} si intende l'insieme delle caratteristiche psichiche e dei comportamentali abituali --- inclinazioni, interessi e passioni --- che definiscono e differenziano ogni individuo, nei vari contesti ed ambienti in cui la condotta umana si sviluppa \cite{corr2009cambridge,sadock2000comprehensive}.

La tradizione di studi psicologici relativi alla personalità è una delle più rilevanti della psicologia contemporanea, un campo in cui si susseguono studi empirici, teorici e storici, volti a comprendere la natura dell'identità personale nel contesto biologico e sociale di sviluppo.
Essa tenta di spiegare le tendenze che sono alla base delle differenze comportamentali, ed ogni gruppo di pensiero tenta di concettualizzare la personalità entro modelli diversi --- adoperando metodi, approcci, obiettivi e modalità d'analisi --- anche molto dissonanti fra loro.

Una significativa parte della psicologia che studia le differenze individuali, analizza e valuta la personalità attraverso specifici test volti ad individuarne i tratti.

\section{Big Five}
\label{sec:big5}

Le teorie della personalità basate sui tratti definiscono la personalità come l'insieme delle caratteristiche che stabiliscono il comportamento di una persona. 

La teoria dei Grandi Cinque (o Big Five) risulta essere uno dei modelli più condivisi e testati, sia a livello teorico che empirico.
McCrae e Costa identificano cinque grandi dimensioni in cui può essere suddivisa la personalità \cite{goldberg1993structure,costa2008revised}:
\begin{itemize}
	\item L'\emph{apertura all'esperienza} o ``openness'' (creativo/curioso vs. coerente/cauto) è intesa come attitudine alla ricerca di stimoli culturali e di pensiero esterni al proprio contesto ordinario.
	Essa riflette il grado di curiosità intellettuale, la creatività o una preferenza per la novità; può inoltre essere percepita come imprevedibilità o mancanza di concentrazione. \\ 
	Individui con un'elevata apertura perseguono l'auto-realizzazione, cercando esperienze intense ed euforiche. Viceversa, coloro che hanno una bassa apertura cercano di ottenere soddisfazione attraverso la perseveranza.
	
	\item La \emph{coscienziosità} o ``conscientiousness'' (organizzato vs. negligente) è una tendenza caratterizzata dall'organizzazione, precisione e affidabilità. Un soggetto contraddistinto da questa attitudine, preferisce un comportamento pianificato piuttosto che spontaneo.\\  
	Spesso l'alta coscienziosità viene percepita come testardaggine e ossessione, mentre la bassa coscienziosità è associata alla flessibilità e alla spontaneità, ma può anche apparire come mancanza di affidabilità.
		
	\item L'\emph{estroversione} o ``extraversion''  (estroverso/energetico vs. solitario/riservato) è intesa come grado di entusiasmo negli atteggiamenti che si adottano e tendenza a cercare la stimolazione in compagnia degli altri.\\ 
	L'alta estroversione è spesso percepita come una ricerca di attenzioni e prepotenza. La bassa estroversione causa una personalità riservata, riflessiva, che può essere avvertita come distaccata.  
	
	\item La \emph{gradevolezza} o ``agreeableness'' (amichevole/compassionevole vs. provocatorio/distaccato) è indicata come quantità e qualità delle relazioni interpersonali che la persona intraprende, orientate al prendersi cura dell'altro. È una tendenza ad essere compassionevoli e collaborativi piuttosto che sospettosi e antagonisti. \\
	L'alta gradevolezza è spesso vista come ingenuità o sottomissione. Le persone con scarsa gradevolezza sono spesso competitive o sfidanti, e possono essere intese come inaffidabili.

	\item Il \emph{nevroticismo} o ``neuroticism'' (sensibile/nervoso vs. sicuro/fiducioso), è una misura di resistenza a stress di tipo psicologico, come l'ansietà e l'irritabilità, ma si riferisce anche al grado di solidità emotiva e di controllo degli impulsi.\\
	Un'alta stabilità si manifesta in una personalità calma che però può essere vista come poco interessante e indifferente. Una bassa stabilità esprime reattività e dinamicità in individui che spesso possono essere percepiti come instabili o insicuri. 
\end{itemize}
Queste dimensioni sono state individuate a partire da studi psico-lessicali, secondo cui le cinque dimensioni corrisponderebbero alle macro-categorie più usate nel linguaggio per descrivere le diversità fra individui.\\
Le regioni cerebrali che codificano i vari tratti di personalità sono spesso collegate alle regioni responsabili della comunicazione verbale e scritta. 

Un grande numero di prove di ricerca hanno supportato il modello a cinque fattori, che sembra essere condiviso a livello interculturale --- Cina, Giappone, Italia, Ungheria, Turchia \cite{triandis2002cultural}.

Le dimensioni di Big Five predicono accuratamente il comportamento e vengono utilizzate sempre più spesso per aiutare i ricercatori a comprendere l'estensione dei disturbi psicologici come ansia e depressione \cite{saulsman2004five}.

Uno dei vantaggi principali di questo approccio è che consente di concentrare l'attenzione solo sulle dimensioni di base piuttosto che studiare centinaia di tratti.
%\begin{table}[]
%	\renewcommand{\arraystretch}{1.5}
%	\centering
%	\begin{tabularx}{\textwidth}{XXXX}
%		\rowcolor[HTML]{C0C0C0} 
%		Dimension & Sample items & Descriptions & Examples of behaviours predicted by trait \tabularnewline 
%		Openness & “I have a vivid imagination”; \newline “I have a rich vocabulary”; \newline“I have excellent ideas.” & A general appreciation for art, emotion, adventure, unusual ideas, imagination,   curiosity, and variety of experience & Individuals who are highly open to experience tend to have distinctive and unconventional decorations in their home. They  are also likely to have books on a wide variety of topics, a diverse music collection, and works of art on display.\tabularnewline  
%		Conscientiousness & “I am always prepared”; \newline“I am exacting in my work”; \newline “I follow a schedule.” & A tendency to show self-discipline, act  dutifully, and aim for achievement & Individuals who are conscientious have a preference for planned rather than spontaneous behaviour. \tabularnewline 
%		Extraversion & “I am the life of the party”;\newline“I feel comfortable around people”;  \newline “I talk to a lot of different people at parties.” & The tendency to experience positive emotions and to seek out stimulation and the company of  others & Extroverts enjoy being with people. In groups they like to talk, assert themselves, and draw attention to themselves. \tabularnewline 
%		Agreeableness & “I am interested in people”; \newline “I feel others’ emotions”; \newline “I make people feel at ease.” & A tendency to be compassionate and  cooperative rather than suspicious and  antagonistic toward others; reflects individual   differences in general concern for social  harmony & Agreeable individuals value getting along with others. They are generally considerate, friendly, generous, helpful, and willing to  compromise their interests with those of others.\tabularnewline 
%		Neuroticism & “I am not usually relaxed”;\newline “I get upset easily”; \newline“I am easily disturbed” & The tendency to experience negative emotions, such as anger, anxiety, or depression; sometimes called “emotional instability” & Those who score high in neuroticism are more likely to interpret  ordinary situations as threatening and minor frustrations as  hopelessly difficult. They may have trouble thinking clearly, making decisions, and coping effectively with stress. \tabularnewline 
%	\end{tabularx}%
%	\caption{The Five Factors of the Five-Factor Model of Personality}
%\label{tab:big5}
%\end{table}


