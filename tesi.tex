%CLASSE DOCUMENTO - LINGUA E DIMENSIONE FONT
\documentclass[11pt]{toptesi}

%%%%%%%%%%%%%%%%%%%%%%%%%%%%%%%%%%%%%%%%%%%%%%%%%%%%%%%%%%%%%%%

% INCLUSIONE PACCHETTI
		
%s\usepackage{fullpage}	
\usepackage{graphicx}
\usepackage[pagestyles]{titlesec}
\usepackage[flushleft]{caption}
\usepackage{setspace}
\usepackage[utf8]{inputenc} %utf8 % lettere accentate da tastiera
\usepackage[italian]{babel} % lingua del documento
\usepackage[T1]{fontenc} % codifica dei font
\usepackage{blindtext}
\usepackage{graphicx,wrapfig}
\usepackage{booktabs}
\usepackage[table,xcdraw]{xcolor}
\usepackage{lmodern}
\usepackage{varioref}
\usepackage{url}
\usepackage{array}
\usepackage{paralist}{\obeyspaces\global\let =\space}
\usepackage{verbatim} 
\usepackage{subfig}
\usepackage{tabularx}
\usepackage{amsmath}
\usepackage{amsfonts}
\usepackage{float}
\usepackage{amssymb}
\usepackage{multicol}
\usepackage{color}
\usepackage{multirow}
\usepackage{listings}
\usepackage[pass]{geometry}
\usepackage[figuresright]{rotating}
\usepackage{algorithm}
\usepackage{algorithmic}
\usepackage{amsmath}
\usepackage[babel]{csquotes}
\usepackage{hyperref}
\usepackage[backend=bibtex,sorting=none]{biblatex}
\usepackage{array}
\usepackage{nameref}
%\newcommand{\cellavuota}{\multicolumn{1}{c|}{}}
%\newcommand{\centra}[1]{\begin{center}#1\end{center}   }

\newcommand{\setfont}[1]{\fontfamily{iwona}\selectfont \scshape #1}
%%%%%%%%%%%%%%%%%%%%%%%%%%%%%%%%%%%%%%%%%%%%%%%%%%%%%%%%%%%%%%%

% CONFIGURAZIONE LINK E RIFERIMENTI
\hypersetup{%
    pdfpagemode={UseOutlines},
    bookmarksopen,
    pdfstartview={FitH},
    colorlinks,
    linkcolor={black}, %COLORE DEI RIFERIMENTI AL TESTO
    citecolor={blue}, %COLORE DEI RIFERIMENTI ALLE CITAZIONI
    urlcolor={blue} %COLORI DEGLI URL
}

%%%%%%%%%%%%%%%%%%%%%%%%%%%%%%%%%%%%%%%%%%%%%%%%%%%%%%%%%%%%%%%

% CONFIGURAZIONE LISTATI/CODICE - CANCELLARE SE NON NECESSARIO
% PYTHON - BIANCO E NERO
\lstset{%
	captionpos=b,
	language=Python,
	basicstyle =\small\ttfamily,
	keywordstyle=\color{black}\bfseries,
	breaklines=true,
	breakatwhitespace=true,
	frame=lines,
	numbers=left,
	numberstyle=\footnotesize,
}

%%%%%%%%%%%%%%%%%%%%%%%%%%%%%%%%%%%%%%%%%%%%%%%%%%%%%%%%%%%%%%%

% FRENCHSPACING VA _SEMPRE_ ABILITATO PER DOCUMENTI IN ITALIANO
\frenchspacing

%%%%%%%%%%%%%%%%%%%%%%%%%%%%%%%%%%%%%%%%%%%%%%%%%%%%%%%%%%%%%%%

%DEFINIZIONE SEZIONI IN NUMERAZIONE ROMANA
%ELENCO DEI LISTATI/CODICI
\makeatletter
\newcommand\listofcodes{%
 \iffrontmatter\else\frontmattertrue\fi
 \if@openright\cleardoublepage\else\clearpage\fi
 % change the meaning of \chapter in a group
 \begingroup\def\chapter##1{\@schapter}
 \phantomsection % for the hyperlink
 \lstlistoflistings 
 \endgroup
} 
\makeatother

%%%%%%%%%%%%%%%%%%%%%%%%%%%%%%%%%%%%%%%%%%%%%%%%%%%%%%%%%%%%%%%

% INFORMAZIONI PDF - PERSONALIZZARE
\pdfinfo{%
  /Title    (Reti neutrali per l’apprendimento dei tratti della personalità dal linguaggio naturale)
  /Author   (Giorgia Adorni)
  /Subject  (Laura Informatica)
  /Keywords (Tesi)
}

%%%%%%%%%%%%%%%%%%%%%%%%%%%%%%%%%%%%%%%%%%%%%%%%%%%%%%%%%%%%%%%


% LISTA DEI CAPITOLI DA INCLUDERE - PERSONALIZZARE
\includeonly{%
frontespizio,%
abstract,%
introduzione,%
contesto, %
neural_network,%
formulazione,
esperimenti,
conclusioni,
chap_quo,%
chap_qua,%
app_a,%
}

%%%%%%%%%%%%%%%%%%%%%%%%%%%%%%%%%%%%%%%%%%%%%%%%%%%%%%%%%%%%%%%


% FILE DI BIBLIOGRAFIA
\bibliography{bibliography} 

% INIZIO DOCUMENTO
\begin{document}
	
% FRONTESPIZIO
\newpage
\pagestyle{empty} % no number
\noindent

\begin{figure}\doublespacing
	\mbox{
				\begin{minipage}{.20\textwidth}
					\includegraphics[height=3.3cm]{images/LogoBicocca.pdf}
				\end{minipage}%
				\quad\quad
				\begin{minipage}[c]{.90\textwidth}
					{Università degli Studi Milano Bicocca}\\
					{  \textbf{Scuola di Scienze}}\\
					{\textbf{Dipartimento di Informatica, Sistemistica e Comunicazione}}\\
					{  \textbf{Corso di Laurea in Informatica}}
				\end{minipage}%
	}

\end{figure}

\begin{center}
	\vspace{35mm}
\doublespacing\textbf{\huge RETI NEURALI PER~L’APPRENDIMENTO DEI TRATTI DELLA PERSONALITÀ DAL LINGUAGGIO NATURALE }\\
	
\end{center}

\vspace{30mm}
\onehalfspacing 

\begin{tabular}{ll}
	\textbf{Relatore: } & {Prof. Stella Fabio Antonio}\\
	\textbf{Co-relatore: } & {Dott. Marelli Marco}
\end{tabular}

\vspace{5mm}

\begin{flushright}\onehalfspacing 
	\textbf{Relazione della prova finale di:}\\
	{Giorgia Adorni}\\
	{Matricola 806787}\end{flushright}

\vspace{25mm}
\begin{center} \textbf{Anno Accademico 2017-2018 }\end{center}


%%%%%%%%%%%%%%%%%%%%%%%%%%%%%%%%%%%%%%%%%%%%%%%%%%%%%%%%%%%%%%%

%INTERLINEA - DEFAULT 1 - NON ESAGERATE, NON SUPERATE MAI 1.3 ;)
%\interlinea{1.2}

%%%%%%%%%%%%%%%%%%%%%%%%%%%%%%%%%%%%%%%%%%%%%%%%%%%%%%%%%%%%%%%
\frontmatter

% DEDICA
% VSPACE - PROPORZIONE USATA PER CENTRATURA VERTICALE DEL TESTO
% FLUSHRIGHT - ALLINEAMENTO ORIZZONTALE A DESTRA
\vspace*{\stretch{1}}
\begin{flushright}
\noindent
\textit{A mio padre. }\\
\textit{Al mio fidanzato e collega. }
\end{flushright}
\vspace*{\stretch{6}}
\cleardoublepage


% CITAZIONE 
% VSPACE - PROPORZIONE USATA PER CENTRATURA VERTICALE DEL TESTO
% FLUSHRIGHT - ALLINEAMENTO ORIZZONTALE A DESTRA
\vspace*{\stretch{1}}
\begin{flushright}
\noindent
Citatemi dicendo che sono stato citato male.
\textit{Groucho Marx}
\end{flushright}

\vspace*{\stretch{6}}
\cleardoublepage

%%%%%%%%%%%%%%%%%%%%%%%%%%%%%%%%%%%%%%%%%%%%%%%%%%%%%%%%%%%%%%%

% RINGRAZIAMENTI - PERSONALIZZARE
\ringraziamenti
Grazie al mio relatore Fabio Stella e ai ragazzi del laboratorio MAD (Models and Algorithms for Data \& text mining).

%%%%%%%%%%%%%%%%%%%%%%%%%%%%%%%%%%%%%%%%%%%%%%%%%%%%%%%%%%%%%%%

% ABSTRACT - PERSONALIZZARE
\sommario

La personalità è considerata come uno degli argomenti di ricerca più influenti in psicologia poiché predittiva di molti esiti consequenziali come la salute mentale e fisica, ed è in grado di spiegare il comportamento umano.
Grazie alla diffusione dei Social Network come mezzo di comunicazione, sta diventando sempre più importante sviluppare modelli che possano leggere automaticamente e con precisione l'essenza di individui basandosi esclusivamente sulla scrittura. 
\\\\
In particolare la convergenza tra scienze sociali e informatiche ha portato i ricercatori a sviluppare approcci automatici per estrarre e studiare le informazioni "nascoste" nei dati testuali presenti in rete.
La natura di questo progetto di tesi è altamente sperimentale, e la motivazione alla base di questo lavoro è presentare delle analisi dettagliate sull'argomento, in quanto allo stato attuale non esistono importanti indagini di questo tipo.
\\\\
Obiettivo è identificare un adeguato spazio semantico che permetta di definire la personalità dell'oggetto a cui un determinato testo si riferisce. Punto di partenza è un dizionario di aggettivi che la letteratura psicologica definisce come marker dei cinque grandi tratti di personalità o Big Five.
\\\\
In questo lavoro siamo partiti dall'implementazione di reti neurali  fully-connected come base per capire come i modelli semplici in deep-learning possano fornire informazioni sulle caratteristiche della personalità nascoste. 
\\\\
Infine utilizziamo una classe di algoritmi distribuzionali inventati nel 2013 da \emph{Tomas Mikolov}, che consistono nell'utilizzo di una rete neurale convoluzionale, che impara in modo non supervisionato i contesti delle parole.
In questo modo costruiamo un embedding in cui sono contenute le informazioni semantiche del testo, ottenendo una sorta di “geometria del significato” in cui i concetti sono tradotti in relazioni lineari.
Con quest'ultimo esperimento ipotizziamo che uno stile di scrittura individuale sia in gran parte accoppiato con i tratti della sua personalità.

%%%%%%%%%%%%%%%%%%%%%%%%%%%%%%%%%%%%%%%%%%%%%%%%%%%%%%%%%%%%%%%
% INDICI 

% INDICE GENERALE
\tableofcontents

% INDICE DELLE FIGURE
\listoffigures

% INDICE DELLE TABELLE
\listoftables

% INDICE DEI CODICI
\listofcodes

%%%%%%%%%%%%%%%%%%%%%%%%%%%%%%%%%%%%%%%%%%%%%%%%%%%%%%%%%%%%%%%
\introduzione
\chapter{Introduzione}
\label{chap:introduzione}
\Blindtext

Data mining: estrazione di significato da grandi quantità di dati
analisi dell'espressione umana che rendo conto da una parte del linguaggio come veicolo di definizione e di affermazione del se
evoluzione della comunicazione verso le nuove forme di dialogo tra persone e comunità

la raccolta di dadi e la necessità di produrre analisi sempre più velocemente ha fatto crescere in questa direzione la ricerca nel campo dell'apprendimento automatico o \emph{machine learning}.
In particolare, nell'ambito del text mining, si sono sviluppate metodologie che consentono ai computer di confrontarsi con il linguaggio umano, di elaborarlo e comprenderlo.


L'obiettivo di questa tesi è descrivere e sfruttare alcune tecniche come il machine learning e Natural Language Processing nell'ambito del ... 


Extroversion was the strongest predictor of leadership emergence — who becomes a leader — and leadership effectiveness — who's successful in a leadership position. But it was a better predictor of emergence than effectiveness.

What's more, when the study authors deconstructed extroversion into distinct parts, they found that dominance and sociability better predicted leadership than extroversion as a whole. This makes sense, the study authors write, "as both sociable and dominant people are more likely to assert themselves in group situations."
Conscientiousness, or a person's tendency to be organized and hard-working, was the second strongest predictor of leadership.
Again, conscientiousness was more closely linked to leader emergence than to leadership effectiveness. The authors write: "[T]he organizing activities of conscientious individuals (e.g. note taking, facilitating processes) may allow such individuals to quickly emerge as leaders.

In business settings, openness to experience is an important predictor of leadership. Justin Sullivan / Getty
Openness to experience was the third strongest predictor of leadership. However, it's worth noting that, in business settings specifically, openness was just as strongly linked to leadership as extroversion.
Neuroticism was not a strong predictor of leadership, meaning that highly neurotic people are not especially likely or unlikely to become leaders.
Agreeableness, or friendliness, was the "least relevant" to leadership of all the traits studied. Interestingly, however, when the researchers looked only at leadership effectiveness, agreeableness was related.




Psicologia e informatica:
I modelli computazionali possono essere utilizzati per modellare i sistemi come una scatola nera, ma possono anche essere usati per informare i modelli di elaborazione delle informazioni che mirano a comprendere la cognizione umana.




Le regioni cerebrali che codificano per vari tratti di personalità sono spesso accoppiate con regioni responsabili della comunicazione verbale e scritta. Inoltre, l'avvento dei social media e una comunità online sempre più connessa rendono sempre più disponibili i dati testuali personalizzati. In questo studio, ipotizziamo che uno stile di scrittura individuale è in gran parte accoppiato con i tratti della sua personalità e presenta un modello di apprendimento profondo per predire il tipo di personalità di Myers Briggs attraverso i dati testuali dei libri. Sviluppare un modello accurato e aprire questa domanda di ricerca avrebbe implicazioni significative nella business intelligence, nell'analisi della compatibilità delle relazioni e in altri campi della sociologia.










\chapter*{Abstract}
\label{Abstract}


Apprendimento della personalità basato sul linguaggio naturale. 
Reti neurali per la previsione dei tipi di personalità di Myers Brigg dagli stili di scrittura.






La personalità è l'essenza che definisce un individuo in quanto guida il modo in cui pensiamo, agiamo e interpretiamo stimoli esterni. 
Nel corso del secolo scorso, gli aspetti della personalità sono stati studiati da molti punti di vista, sia attraverso l'analisi delle relazioni interpersonali, delle dinamiche di gruppo e dei social network, sia attraverso le opere nelle neuroscienze che rivelano le basi biologiche dei tratti della personalità.


Il testo e la scrittura sono diventati un mezzo di comunicazione importante nell'era digitale. L'uso crescente dei social media e il fenomeno delle recensioni online hanno favorito un rapido aumento dei dati testuali digitali su cui vengono effettuate svariati tipi di analisi.
In particolare la convergenza tra scienze sociali (psicologiche) e informatiche hanno portato i ricercatori a sviluppare metodi automatizzati (approcci automatici) per estrarre e studiare le informazioni digitali contenute nel materiale testuale per prevedere i tratti della personalità.
La maggior parte degli attuali studi automatici di rilevamento della personalità si sono concentrati sul modello di personalità Big 5 come quadro per studiare le caratteristiche intrinseche dell'essere umano. 


Partendo da un dizionario di aggettivi che la letteratura psicologica definisce come marker dei cinque grandi tratti di personalità (Big Five), si vuole identificare un adeguato spazio semantico che permetta di definire la personalità dell’oggetto target a cui un determinato testo si riferisce. I dati che verranno utilizzati per definire lo spazio semantico e testare la sua funzionalità sono messi a disposizione da Yelp Dataset Challenge, che contiene 5˙200˙000 reviews relative a 174˙000 businesses di 11 aree metropolitane nel mondo. 




Obiettivo è la progettazione e lo sviluppo di un modello computazionale per l'apprendimento della personalità a partire dal linguaggio naturale.  


In questo studio, ipotizziamo che uno stile di scrittura individuale è in gran parte accoppiato con i tratti della sua personalità e presenta un modello di apprendimento profondo per predire il tipo di personalità di Myers Briggs attraverso i dati testuali dei libri. Sviluppare un modello accurato e aprire questa domanda di ricerca avrebbe implicazioni significative nella business intelligence, nell'analisi della compatibilità delle relazioni e in altri campi della sociologia.



La personalità è considerata uno degli argomenti di ricerca più influenti in psicologia perché è predittiva di molti esiti consequenziali come la salute mentale e fisica, la qualità delle relazioni interpersonali, l'adeguamento alla carriera e la soddisfazione, le prestazioni sul posto di lavoro e il benessere generale.

È ampiamente noto che i tratti della personalità come l'extraversione, la coscienziosità e il nevroticismo sono relativamente coerenti per tutta la vita. Tuttavia, i modi in cui i nostri comportamenti sono espressi attraverso le parole e l'azione non sono sempre determinati dai tratti soggettivi della personalità e dagli impulsi da soli.

Molte decisioni importanti, le dinamiche sociali e le decisioni politiche si basano anche sulla valutazione della personalità di un individuo con il quale non si è interagito molto personalmente. 

Ma leggere solo i comportamenti di altre persone non è sufficiente per fare previsioni accurate della loro personalità. Il compito diventa ancora più difficile quando si tenta di formulare giudizi basati solo sulla comunicazione scritta. 

Poiché il mondo si basa molto più sulla comunicazione basata sul testo rispetto alle interazioni faccia a faccia, sta diventando sempre più importante sviluppare modelli che possano leggere automaticamente e con precisione l'essenza di altri individui basandosi esclusivamente sulla scrittura. 

Fortunatamente, studi in neuroscienza hanno rivelato una mappatura vicina delle regioni del cervello responsabili dei tratti della personalità come l'extraversione e il nevroticismo, nonché quelli che sono legati alla comunicazione scritta.

I precedenti modelli di previsione della personalità si sono concentrati sull'applicazione di tecniche generali di apprendimento automatico e reti neurali per predire i tratti di personalità del Big Five di openness, conscientiousness, extraversion, agreeableness e neuroticism dai post sui social media. 

Gli studi che si focalizzano sui tratti del Big Five tendono a dare un tratto alla figura di un individuo.


In questo lavoro, abbiamo esplorato una varietà di metodi per affrontare il problema della predizione della personalità. 

Abbiamo iniziato costruendo manualmente un vasto corpus di brani tratti da romanzi famosi con autori di tipi MBTI. 

Per valutare la difficoltà di identificare gli MBTI dal testo, abbiamo raggruppato segmenti di testo basati su somiglianze di incorporamento di parole per determinare se esistesse una distribuzione non uniforme di tipi di personalità. Ciò fornisce una buona cornice iniziale di riferimento per capire quanto siano sottili i tratti della personalità quando sono nascosti nei dati scritti. 

Abbiamo poi implementato un sacco di reti neurali feed-forward come base per capire come i modelli semplici in deep-learning possano fornire informazioni sulle caratteristiche della personalità nascoste. 

Infine, ci addentriamo in una rete neuronale ricorrente basata sulla memoria a lungo termine più complessa e miriamo a costruire un sistema più generalizzabile che possa incorporare il significato della scrittura per determinare i tipi di personalità generali.



tratti di personalità Big 5.

I risultati delle analisi mostrano che il potere predittivo delle impronte digitali sui tratti della personalità è in linea con il 
Nel complesso, i nostri risultati indicano che la precisione delle previsioni è coerente tra i tratti Big 5 e che l'accuratezza migliora quando le analisi includono dati demografici e diversi tipi di impronte digitali.



struttura di personalità per guidare la nostra comprensione e rivelare il ruolo delle parole nel descrivere le caratteristiche di un utente. 
Questo studio preliminare ha rivelato come le percezioni del pubblico riguardo a parole specifiche possano aiutarci a rilevare la personalità. 

Come prima fase del nostro studio, questo esperimento si concentra sulla raccolta delle percezioni generali dei malesi verso 52 aggettivi inglesi categorizzazione delle parole sotto tratti PEN. 
La valutazione fornisce l'analisi necessaria che potrebbe aiutare la nostra ricerca principale che si concentra su rilevazioni automatiche della personalità. 










Abbiamo iniziato costruendo manualmente un vasto corpus di brani tratti da romanzi famosi con autori di tipi MBTI. 

Per valutare la difficoltà di identificare gli MBTI dal testo, abbiamo raggruppato segmenti di testo basati su somiglianze di incorporamento di parole per determinare se esistesse una distribuzione non uniforme di tipi di personalità. Ciò fornisce una buona cornice iniziale di riferimento per capire quanto siano sottili i tratti della personalità quando sono nascosti nei dati scritti. 



I risultati delle analisi mostrano che il potere predittivo delle impronte digitali sui tratti della personalità è in linea con il 
Nel complesso, i nostri risultati indicano che la precisione delle previsioni è coerente tra i tratti Big 5 e che l'accuratezza migliora quando le analisi includono dati demografici e diversi tipi di impronte digitali.


struttura di personalità per guidare la nostra comprensione e rivelare il ruolo delle parole nel descrivere le caratteristiche di un utente. 
Questo studio preliminare ha rivelato come le percezioni del pubblico riguardo a parole specifiche possano aiutarci a rilevare la personalità. 

Come prima fase del nostro studio, questo esperimento si concentra sulla raccolta delle percezioni generali dei malesi verso 52 aggettivi inglesi categorizzazione delle parole sotto tratti PEN. 
La valutazione fornisce l'analisi necessaria che potrebbe aiutare la nostra ricerca principale che si concentra su rilevazioni automatiche della personalità. 






% INCLUSIONE FILE CAPITOLI - TENERE COERENTE CON LISTA IN ALTO

\mainmatter
%\chapter{Introduzione}
\label{chap:introduzione}
\Blindtext

Data mining: estrazione di significato da grandi quantità di dati
analisi dell'espressione umana che rendo conto da una parte del linguaggio come veicolo di definizione e di affermazione del se
evoluzione della comunicazione verso le nuove forme di dialogo tra persone e comunità

la raccolta di dadi e la necessità di produrre analisi sempre più velocemente ha fatto crescere in questa direzione la ricerca nel campo dell'apprendimento automatico o \emph{machine learning}.
In particolare, nell'ambito del text mining, si sono sviluppate metodologie che consentono ai computer di confrontarsi con il linguaggio umano, di elaborarlo e comprenderlo.


L'obiettivo di questa tesi è descrivere e sfruttare alcune tecniche come il machine learning e Natural Language Processing nell'ambito del ... 


Extroversion was the strongest predictor of leadership emergence — who becomes a leader — and leadership effectiveness — who's successful in a leadership position. But it was a better predictor of emergence than effectiveness.

What's more, when the study authors deconstructed extroversion into distinct parts, they found that dominance and sociability better predicted leadership than extroversion as a whole. This makes sense, the study authors write, "as both sociable and dominant people are more likely to assert themselves in group situations."
Conscientiousness, or a person's tendency to be organized and hard-working, was the second strongest predictor of leadership.
Again, conscientiousness was more closely linked to leader emergence than to leadership effectiveness. The authors write: "[T]he organizing activities of conscientious individuals (e.g. note taking, facilitating processes) may allow such individuals to quickly emerge as leaders.

In business settings, openness to experience is an important predictor of leadership. Justin Sullivan / Getty
Openness to experience was the third strongest predictor of leadership. However, it's worth noting that, in business settings specifically, openness was just as strongly linked to leadership as extroversion.
Neuroticism was not a strong predictor of leadership, meaning that highly neurotic people are not especially likely or unlikely to become leaders.
Agreeableness, or friendliness, was the "least relevant" to leadership of all the traits studied. Interestingly, however, when the researchers looked only at leadership effectiveness, agreeableness was related.




Psicologia e informatica:
I modelli computazionali possono essere utilizzati per modellare i sistemi come una scatola nera, ma possono anche essere usati per informare i modelli di elaborazione delle informazioni che mirano a comprendere la cognizione umana.




Le regioni cerebrali che codificano per vari tratti di personalità sono spesso accoppiate con regioni responsabili della comunicazione verbale e scritta. Inoltre, l'avvento dei social media e una comunità online sempre più connessa rendono sempre più disponibili i dati testuali personalizzati. In questo studio, ipotizziamo che uno stile di scrittura individuale è in gran parte accoppiato con i tratti della sua personalità e presenta un modello di apprendimento profondo per predire il tipo di personalità di Myers Briggs attraverso i dati testuali dei libri. Sviluppare un modello accurato e aprire questa domanda di ricerca avrebbe implicazioni significative nella business intelligence, nell'analisi della compatibilità delle relazioni e in altri campi della sociologia.










\chapter*{Abstract}
\label{Abstract}


Apprendimento della personalità basato sul linguaggio naturale. 
Reti neurali per la previsione dei tipi di personalità di Myers Brigg dagli stili di scrittura.






La personalità è l'essenza che definisce un individuo in quanto guida il modo in cui pensiamo, agiamo e interpretiamo stimoli esterni. 
Nel corso del secolo scorso, gli aspetti della personalità sono stati studiati da molti punti di vista, sia attraverso l'analisi delle relazioni interpersonali, delle dinamiche di gruppo e dei social network, sia attraverso le opere nelle neuroscienze che rivelano le basi biologiche dei tratti della personalità.


Il testo e la scrittura sono diventati un mezzo di comunicazione importante nell'era digitale. L'uso crescente dei social media e il fenomeno delle recensioni online hanno favorito un rapido aumento dei dati testuali digitali su cui vengono effettuate svariati tipi di analisi.
In particolare la convergenza tra scienze sociali (psicologiche) e informatiche hanno portato i ricercatori a sviluppare metodi automatizzati (approcci automatici) per estrarre e studiare le informazioni digitali contenute nel materiale testuale per prevedere i tratti della personalità.
La maggior parte degli attuali studi automatici di rilevamento della personalità si sono concentrati sul modello di personalità Big 5 come quadro per studiare le caratteristiche intrinseche dell'essere umano. 


Partendo da un dizionario di aggettivi che la letteratura psicologica definisce come marker dei cinque grandi tratti di personalità (Big Five), si vuole identificare un adeguato spazio semantico che permetta di definire la personalità dell’oggetto target a cui un determinato testo si riferisce. I dati che verranno utilizzati per definire lo spazio semantico e testare la sua funzionalità sono messi a disposizione da Yelp Dataset Challenge, che contiene 5˙200˙000 reviews relative a 174˙000 businesses di 11 aree metropolitane nel mondo. 




Obiettivo è la progettazione e lo sviluppo di un modello computazionale per l'apprendimento della personalità a partire dal linguaggio naturale.  


In questo studio, ipotizziamo che uno stile di scrittura individuale è in gran parte accoppiato con i tratti della sua personalità e presenta un modello di apprendimento profondo per predire il tipo di personalità di Myers Briggs attraverso i dati testuali dei libri. Sviluppare un modello accurato e aprire questa domanda di ricerca avrebbe implicazioni significative nella business intelligence, nell'analisi della compatibilità delle relazioni e in altri campi della sociologia.



La personalità è considerata uno degli argomenti di ricerca più influenti in psicologia perché è predittiva di molti esiti consequenziali come la salute mentale e fisica, la qualità delle relazioni interpersonali, l'adeguamento alla carriera e la soddisfazione, le prestazioni sul posto di lavoro e il benessere generale.

È ampiamente noto che i tratti della personalità come l'extraversione, la coscienziosità e il nevroticismo sono relativamente coerenti per tutta la vita. Tuttavia, i modi in cui i nostri comportamenti sono espressi attraverso le parole e l'azione non sono sempre determinati dai tratti soggettivi della personalità e dagli impulsi da soli.

Molte decisioni importanti, le dinamiche sociali e le decisioni politiche si basano anche sulla valutazione della personalità di un individuo con il quale non si è interagito molto personalmente. 

Ma leggere solo i comportamenti di altre persone non è sufficiente per fare previsioni accurate della loro personalità. Il compito diventa ancora più difficile quando si tenta di formulare giudizi basati solo sulla comunicazione scritta. 

Poiché il mondo si basa molto più sulla comunicazione basata sul testo rispetto alle interazioni faccia a faccia, sta diventando sempre più importante sviluppare modelli che possano leggere automaticamente e con precisione l'essenza di altri individui basandosi esclusivamente sulla scrittura. 

Fortunatamente, studi in neuroscienza hanno rivelato una mappatura vicina delle regioni del cervello responsabili dei tratti della personalità come l'extraversione e il nevroticismo, nonché quelli che sono legati alla comunicazione scritta.

I precedenti modelli di previsione della personalità si sono concentrati sull'applicazione di tecniche generali di apprendimento automatico e reti neurali per predire i tratti di personalità del Big Five di openness, conscientiousness, extraversion, agreeableness e neuroticism dai post sui social media. 

Gli studi che si focalizzano sui tratti del Big Five tendono a dare un tratto alla figura di un individuo.


In questo lavoro, abbiamo esplorato una varietà di metodi per affrontare il problema della predizione della personalità. 

Abbiamo iniziato costruendo manualmente un vasto corpus di brani tratti da romanzi famosi con autori di tipi MBTI. 

Per valutare la difficoltà di identificare gli MBTI dal testo, abbiamo raggruppato segmenti di testo basati su somiglianze di incorporamento di parole per determinare se esistesse una distribuzione non uniforme di tipi di personalità. Ciò fornisce una buona cornice iniziale di riferimento per capire quanto siano sottili i tratti della personalità quando sono nascosti nei dati scritti. 

Abbiamo poi implementato un sacco di reti neurali feed-forward come base per capire come i modelli semplici in deep-learning possano fornire informazioni sulle caratteristiche della personalità nascoste. 

Infine, ci addentriamo in una rete neuronale ricorrente basata sulla memoria a lungo termine più complessa e miriamo a costruire un sistema più generalizzabile che possa incorporare il significato della scrittura per determinare i tipi di personalità generali.



tratti di personalità Big 5.

I risultati delle analisi mostrano che il potere predittivo delle impronte digitali sui tratti della personalità è in linea con il 
Nel complesso, i nostri risultati indicano che la precisione delle previsioni è coerente tra i tratti Big 5 e che l'accuratezza migliora quando le analisi includono dati demografici e diversi tipi di impronte digitali.



struttura di personalità per guidare la nostra comprensione e rivelare il ruolo delle parole nel descrivere le caratteristiche di un utente. 
Questo studio preliminare ha rivelato come le percezioni del pubblico riguardo a parole specifiche possano aiutarci a rilevare la personalità. 

Come prima fase del nostro studio, questo esperimento si concentra sulla raccolta delle percezioni generali dei malesi verso 52 aggettivi inglesi categorizzazione delle parole sotto tratti PEN. 
La valutazione fornisce l'analisi necessaria che potrebbe aiutare la nostra ricerca principale che si concentra su rilevazioni automatiche della personalità. 










Abbiamo iniziato costruendo manualmente un vasto corpus di brani tratti da romanzi famosi con autori di tipi MBTI. 

Per valutare la difficoltà di identificare gli MBTI dal testo, abbiamo raggruppato segmenti di testo basati su somiglianze di incorporamento di parole per determinare se esistesse una distribuzione non uniforme di tipi di personalità. Ciò fornisce una buona cornice iniziale di riferimento per capire quanto siano sottili i tratti della personalità quando sono nascosti nei dati scritti. 



I risultati delle analisi mostrano che il potere predittivo delle impronte digitali sui tratti della personalità è in linea con il 
Nel complesso, i nostri risultati indicano che la precisione delle previsioni è coerente tra i tratti Big 5 e che l'accuratezza migliora quando le analisi includono dati demografici e diversi tipi di impronte digitali.


struttura di personalità per guidare la nostra comprensione e rivelare il ruolo delle parole nel descrivere le caratteristiche di un utente. 
Questo studio preliminare ha rivelato come le percezioni del pubblico riguardo a parole specifiche possano aiutarci a rilevare la personalità. 

Come prima fase del nostro studio, questo esperimento si concentra sulla raccolta delle percezioni generali dei malesi verso 52 aggettivi inglesi categorizzazione delle parole sotto tratti PEN. 
La valutazione fornisce l'analisi necessaria che potrebbe aiutare la nostra ricerca principale che si concentra su rilevazioni automatiche della personalità. 





\chapter{Contesto}
\label{chap:contesto}

Con il termine \emph{personalità} si intende l'insieme delle caratteristiche psichiche e dei comportamentali abituali --- inclinazioni, interessi e passioni --- che definiscono e differenziano ogni individuo, nei vari contesti ed ambienti in cui la condotta umana si sviluppa \cite{corr2009cambridge,sadock2000comprehensive}.

La tradizione di studi psicologici relativi alla personalità è una delle più rilevanti della psicologia contemporanea, un campo in cui si susseguono studi empirici, teorici e storici, tesi a comprendere la natura dell'identità personale nel contesto biologico e sociale di sviluppo.
Essa tenta di spiegare le tendenze che sono alla base delle differenze comportamentali, ed ogni gruppo di pensiero tenta di concettualizzare la personalità entro modelli diversi, adoperando metodi, approcci, obiettivi e modalità d'analisi, anche molto dissonanti fra loro.

Una significativa parte della psicologia delle differenze individuali, analizza e valuta la personalità attraverso test volti ad individuarne i tratti.

\section{Big Five}
\label{sec:big5}

Le teorie della personalità basate sui tratti definiscono la personalità come l'insieme delle caratteristiche che stabiliscono il comportamento di una persona. 

La teoria dei Grandi Cinque (o Big Five) risulta essere uno dei modelli più condivisi e testati, sia a livello teorico che empirico.
McCrae e Costa identificano cinque grandi dimensioni in cui può essere suddivisa la personalità \cite{goldberg1993structure,costa2008revised}:
\begin{itemize}
	\item L'\emph{apertura all'esperienza} o ``openness'' (creativo/curioso vs. coerente/cauto) è intesa come attitudine alla ricerca di stimoli culturali e di pensiero esterni al proprio contesto ordinario.
	Essa riflette il grado di curiosità intellettuale, la creatività o una preferenza per la novità e può inoltre essere percepita come imprevedibilità o mancanza di concentrazione. \\ 
	Individui con un'elevata apertura perseguono l'auto-realizzazione, cercando esperienze intense ed euforiche. Viceversa, coloro che hanno una bassa apertura cercano di ottenere soddisfazione attraverso la perseveranza.
	
	\item La \emph{coscienziosità} o ``conscientiousness'' (organizzato vs. negligente) è una tendenza caratterizzata dall'organizzazione, precisione e affidabilità. Un soggetto contraddistinto da questa attitudine, preferisce un comportamento pianificato piuttosto che spontaneo.\\  
	Spesso l'alta coscienziosità viene percepita come testardaggine e ossessione, mentre la bassa coscienziosità è associata alla flessibilità e alla spontaneità, ma può anche apparire come mancanza di affidabilità.
		
	\item L'\emph{estroversione} o ``extraversion''  (estroverso/energetico vs. solitario/riservato) è intesa come grado di entusiasmo nelle atteggiamenti che si adottano e tendenza a cercare la stimolazione in compagnia degli altri.\\ 
	L'alta estroversione è spesso percepita come una ricerca di attenzioni e prepotenza. La bassa estroversione causa una personalità riservata, riflessiva, che può essere avvertita come distaccata.  
	
	\item La \emph{gradevolezza} o ``agreeableness'' (amichevole/compassionevole vs. provocatorio/distaccato) è indicata come quantità e qualità delle relazioni interpersonali che la persona intraprende, orientate al prendersi cura dell'altro. È una tendenza ad essere compassionevoli e collaborativi piuttosto che sospettosi e antagonisti. \\
	L'alta gradevolezza è spesso vista come ingenuità o sottomissione. Le persone con scarsa gradevolezza sono spesso competitive o sfidanti, e possono essere intese come inaffidabili.

	\item Il \emph{nevroticismo} o ``neuroticism'' (sensibile/nervoso vs. sicuro/fiducioso), è una misura di resistenza a stress di tipo psicologico, come l'ansietà e l'irritabilità, ma si riferisce anche al grado di solidità emotiva e di controllo degli impulsi.\\
	Un'alta stabilità si manifesta in una personalità calma che però può essere vista come poco interessante e indifferente. Una bassa stabilità esprime reattività e dinamicità in individui che spesso possono essere percepiti come instabili o insicuri. 
\end{itemize}
Queste dimensioni sono state individuate a partire da studi psico-lessicali, secondo cui le cinque dimensioni corrisponderebbero alle macro-categorie più usate nel linguaggio per descrivere le diversità tra individui.\\
Le regioni cerebrali che codificano i vari tratti di personalità sono spesso collegate alle regioni responsabili della comunicazione verbale e scritta. 

Un grande numero di prove di ricerca hanno supportato il modello a cinque fattori, che sembra essere condiviso a livello interculturale --- Cina, Giappone, Italia, Ungheria, Turchia \cite{triandis2002cultural}.

Le dimensioni di Big Five predicono accuratamente il comportamento e vengono utilizzate sempre più spesso per aiutare i ricercatori a comprendere l'estensione dei disturbi psicologici come ansia e depressione \cite{saulsman2004five}.

Uno dei vantaggi principali di questo approccio è che consente di concentrare l'attenzione solo sulle dimensioni di base piuttosto che studiare centinaia di tratti.
%\begin{table}[]
%	\renewcommand{\arraystretch}{1.5}
%	\centering
%	\begin{tabularx}{\textwidth}{XXXX}
%		\rowcolor[HTML]{C0C0C0} 
%		Dimension & Sample items & Descriptions & Examples of behaviours predicted by trait \tabularnewline 
%		Openness & “I have a vivid imagination”; \newline “I have a rich vocabulary”; \newline“I have excellent ideas.” & A general appreciation for art, emotion, adventure, unusual ideas, imagination,   curiosity, and variety of experience & Individuals who are highly open to experience tend to have distinctive and unconventional decorations in their home. They  are also likely to have books on a wide variety of topics, a diverse music collection, and works of art on display.\tabularnewline  
%		Conscientiousness & “I am always prepared”; \newline“I am exacting in my work”; \newline “I follow a schedule.” & A tendency to show self-discipline, act  dutifully, and aim for achievement & Individuals who are conscientious have a preference for planned rather than spontaneous behaviour. \tabularnewline 
%		Extraversion & “I am the life of the party”;\newline“I feel comfortable around people”;  \newline “I talk to a lot of different people at parties.” & The tendency to experience positive emotions and to seek out stimulation and the company of  others & Extroverts enjoy being with people. In groups they like to talk, assert themselves, and draw attention to themselves. \tabularnewline 
%		Agreeableness & “I am interested in people”; \newline “I feel others’ emotions”; \newline “I make people feel at ease.” & A tendency to be compassionate and  cooperative rather than suspicious and  antagonistic toward others; reflects individual   differences in general concern for social  harmony & Agreeable individuals value getting along with others. They are generally considerate, friendly, generous, helpful, and willing to  compromise their interests with those of others.\tabularnewline 
%		Neuroticism & “I am not usually relaxed”;\newline “I get upset easily”; \newline“I am easily disturbed” & The tendency to experience negative emotions, such as anger, anxiety, or depression; sometimes called “emotional instability” & Those who score high in neuroticism are more likely to interpret  ordinary situations as threatening and minor frustrations as  hopelessly difficult. They may have trouble thinking clearly, making decisions, and coping effectively with stress. \tabularnewline 
%	\end{tabularx}%
%	\caption{The Five Factors of the Five-Factor Model of Personality}
%\label{tab:big5}
%\end{table}



\chapter{Reti Neurali}
\label{chap:RetiNeurali}

Nel campo dell'apprendimento automatico (o machine learning), una rete neurale artificiale (\textbf{ANN} "Artificial Neural Network") è un modello matematico basato sulla semplificazione delle reti neurali biologiche \cite{samuel1959some}. \\
Una rete neurale può essere considerata come un sistema dinamico avente la topologia di un grafo orientato con nodi (i neuroni artificiali) ed archi (pesi sinaptici).
Questo modello è composto da un gruppo di interconnessioni di informazioni costituite da neuroni artificiali che modellano i neuroni in un cervello biologico e processi che utilizzano un approccio di connessionismo di calcolo.
Ogni connessione, come le sinapsi di un cervello biologico, può trasmettere un segnale da un neurone artificiale a un altro, i quali sono tipicamente aggregati in strati. 
Gli stimoli vengono ricevuti da un livello di nodi d'ingresso detto unità di elaborazione. Il neurone artificiale che riceve il segnale può elaborarlo e quindi trasmetterlo ad altri neuroni artificiali collegati ad esso.\\
Poiché nell'implementazione di ANN tradizionali i segnali vengono rappresentati da numeri reali, possiamo considerare questa elaborazione come una moltiplicazione tra gli ingressi e un valore detto peso. Il risultato di queste operazioni viene sommato e se la somma supera un certa soglia il neurone si attiva, attivando la sua uscita.
Il peso serve a quantificare l'importanza dell'efficacia sinaptica della linea di ingresso.		\\
Sono strutture non-lineari di dati statistici e vengono utilizzate per simulare relazioni complesse tra ingressi ed uscite che altre funzioni analitiche non sono in grado di rappresentare.\\

\begin{figure}
	\centering
	{\includegraphics[width=.65\textwidth]{images/ArtificialNeuronModel}} \quad
	\caption{Artificial Neuron Model}
	\label{fig:Artificial Neuron Model}
\end{figure}

\section{Definizioni}
\label{sec:def}
Le reti neurali possono essere viste come semplici modelli matematici che definiscono una funzione $f:X\rightarrow Y$. \\
La funzione di rete di un neurone $f(x)$ è definita come una composizione di altre funzioni $g_i(x)$, che possono a loro volta essere scomposte in altre funzioni. Questo può essere rappresentato come una struttura di reti.\\
Una rappresentazione ampiamente utilizzata è la \emph{somma ponderata non lineare}, dove 
\begin{equation}
f(x)=K \bigg( \sum_{i}w_ig_i(x)\bigg)
\end{equation}
dove $k$ è una funzione predefinita, detta anche \emph{funzione di attivazione}.
\\\\
Ad ogni input $x_i$ è associato un peso $w_i$ con valore positivo o negativo per eccitare o inibire il neurone. Il bias varia secondo la propensione del neurone ad attivarsi, per variare la soglia di attivazione del neurone.
Secondo l'algoritmo, dopo il caricamento dei valori di input e i pesi relativi, si prosegue calcolando la somma dei valori di input pesata con i relativi pesi. In seguito viene calcolato il valore della funzione di attivazione $K$ con il risultato della somma pesata.
L'output del neurone $y$ è il risultato ottenuto dalla funzione di attivazione
\begin{equation}
y(x)=K\bigg( \sum_{i=1}^{d}w_ix_i+w_0\bigg) = \overline{w}^T\overline{x}
\end{equation}

\subsection{Funzione di attivazione}
\label{subsec:fattivazione}
La funzione di attivazione di un nodo determina la risposta di quel neurone.\\
Solo le funzioni di attivazione non lineare consentono a tali reti di calcolare problemi non banali utilizzando un  numero limitato di nodi.\\\\
Utilizzando una funzione di attivazione Sigmoide normalizzabile, il modello realistico rimane a zero fino a quando viene ricevuta la corrente di ingresso, a quel punto la frequenza di accensione aumenta rapidamente all'inizio, ma gradualmente si avvicina a un asintoto con una frequenza di attivazione del 100\% \cite{han1995influence}.

\begin{equation}
f (x) = \frac{1}{1+e^{-x}}
\end{equation}
$\\$
Un'altra delle funzioni di attivazione più utilizzate è la ReLU (Rectified Linear Unit) \cite{nair2010rectified}, definita dalla seguente 
\begin{equation}
f (x) = max(0, x)
\end{equation}
Questa funzione semplicemente setta a zero tutti i valori negativi, mentre ritorna lo stesso valore per quelli positivi.
La ReLU viene utilizzata in quanto rende più rapido ridurre l’accuratezza ed inoltre risolve parzialmente il problema del "vanishing gradient" che porta i layer più in profondità della rete ad imparare troppo lentamente.


\begin{figure}
	\centering
	\subfloat[][\emph{Sigmoide}]
	{\includegraphics[width=.45\textwidth]{images/signmoid2}}
	\quad
	\subfloat[][\emph{ReLU}]
	{\includegraphics[width=.45\textwidth]{images/relu2}} 
	
	\caption{Funzioni di attivazione}
	\label{fig:subfig}
\end{figure}



\section{Apprendimento}
\label{sec:apprendimento}
Per insegnare alla rete a risolvere il problema occorre un periodo di apprendimento in cui vengono sfruttate una serie di osservazioni per trovare ciò che risolvere il compito in maniera ottimale. \\
Questo comporta la definizione di una \emph{funzione di costo} in grado di misurare la distanza tra una soluzione particolare ed una ottimale per il problema. Gli algoritmi di apprendimento cercano una funzione che abbia il minor costo possibile.\\
La conoscenza estratta dalla rete viene memorizzata sui pesi che vengono modificati sfruttando tecniche di ottimizzazione. \\\\
La funzione di errore è l'espressione della differenza/scostamento tra l'output della rete $y$ e l'output desiderato $y'$ nell'apprendimento
\begin{equation}
E(w)=\sum_{i=1}^{c}((y_i-y'_i)^2)
\end{equation}


{\color{blue}
	\subsection{Algoritmo di back propagation}
	\label{subsec:backprop}}

\subsection{Paradigmi di apprendimento}
\label{subsec:Paradigmi di apprendimento}

Gli algoritmi di apprendimento sono principalmente suddivisi in due categorie:
\begin{itemize}
	\item[\bfseries supervisionato] --- alla rete viene presentato un training set preparato da un "insegnante esterno", composto da coppie significative di valori (input, output atteso/desiderato).\\
	Quando alla rete neurale viene fornito l'input dall'ambiente, l'insegnante è in grado di calcolare/restituire l'output desiderato corrispondente addestrando la rete mediante un algoritmo (tipicamente quello di back propagation). 
	Durante questo procedimento viene calcolo l'errore che la rete commette, necessario per farle comprendere di quanto sbaglia ed è dato dalla differenza tra l'output e l'output atteso. La rete modifica i propri pesi in base all'errore commesso, che rappresenta l'azione ottima per l'input corrente, cercando di minimizzarlo.\\
	In questo moto la rete impara a riconoscere la relazione incognita che lega le variabili di ingresso e uscita, in modo da prevedere il valore di output per qualsiasi valore di ingresso, basandosi solo su una limitata casistica di corrispondenza (coppie input-output).
	
	\item[\bfseries non supervisionato] --- alla rete vengono presentati solo i valori di input, mentre non sono messe a disposizione le informazioni di ritorno dell'ambiente sui valori obiettivo che si vogliono ottenere in risposta o riguardo la correttezza dell'output fornito. \\
	La rete individua da sola pattern, caratteristiche, similarità e regolarità statistiche nei dati di input, acquisendo la capacità di dividerli in cluster rappresentativi in modo da sviluppare delle rappresentazioni interne, senza usare confronti con output noti.\\ 
	In questo caso gli algoritmi che modificano i pesi della rete fanno riferimento solo ai dati contenuti nelle variabili di ingresso.\\
	Questo è un tipo di apprendimento autonomo e non c'è controllo esterno sull'errore. Adatto per ottimizzare risorse e se non si conoscono a priori i gruppi in cui dividere l'input.
\end{itemize}


%\subsection{Tasso di apprendimento}
%\label{subsec:learnrate}
%modificando i valori dei pesi dopo ogni presentazione di una configurazione dell'input invece che dopo la presentazione di un ciclo completo di configurazioni, realizza una discesa nello spazio dei pesi che non è necessariamente la più ripida. È comunque corretto approssimare tale discesa alla più ripida purché ad ogni passo del processo iterativo di apprendimento i pesi non vengano modificati di quantità troppo grandi, fissando un tasso di apprendimento $\mu$ piccolo.\\
%Il processo di discesa del gradiente può essere estremamente lento se il valore del tasso di apprendimento $\mu$ è troppo piccolo; d'altra parte, valori troppo elevati di  $\mu$ introducono fenomeni di oscillazioni durante la discesa. Il problema delle oscillazioni è fortemente legato alla presenza di valli profonde con pendenza quasi nulla sul fondo della superficie della funzione errore; in questi casi il vettore dei pesi ad ogni aggiornamento presenta una componente diretta verso il punto di minimo della funzione errore, ma anche una componente che lo devia da tale direzione e che è la causa di oscillazioni che hanno direzione opposta in aggiornamenti consecutivi. \\
%


\section{Development and test set }
\label{sec:DevelopmentAndTestSet}
Per misurare le prestazioni di una rete neurale dopo la fase di apprendimento, viene creato un test set formato da coppie non utilizzate per il training e validation set.\\
Vengono generalmente definiti: 
\begin{itemize}
	\item Training set – Sul quale viene eseguito l'algoritmo di apprendimento.
	\item Dev (development) set – Viene utilizzato per regolare i parametri, selezionare le features e prendere decisioni per quanto riguarda l'algoritmo di apprendimento. Talvolta viene anche chiamato set di hold-out / cross  validation. (convalida incrociata)
	\item Test set – si utilizza per valutare le performance/prestazioni dell'algoritmo, ma non per prendere decisioni su quale algoritmo di apprendimento o parametri utilizzare. 
\end{itemize}

Una volta definiti i set di development e test, ci si concentrerà sul miglioramento delle prestazioni del development set. \\
Generalmente la dimensione di del test set è un terzo del training set, ed è composto da input critici su cui la risposta della rete deve essere buona.
Questo funziona bene quando sono messi a disposizione un numero limitato di esempi, ma nell'era dei big data, dove i problemi di apprendimento automatico consistono di più di un miliardo di esempi, la frazione di dati allocati agli insiemi di sviluppo / test è ridotta, nonostante il valore assoluto di esempi sia aumentato/maggiore.\\
Vengono utilizzate diverse tecniche statistiche per valutare la bontà nella rete. Normalmente si accetta una rete se sul test set viene mostrato un errore inferiore al 20-25\%.



\subsection{Evaluation metric}
\label{subsec:EvaluationMetric}
La \emph{precisione} di classificazione è un esempio di una metrica di valutazione di un singolo numero: si esegue il classificatore sul set di sviluppo (o set di test) e si ottiene un numero singolo su quale frazione di esempi è stata classificata correttamente. Secondo questa metrica, se il classificatore A ottiene l'accuratezza del 97\%, e il classificatore B ottiene l'accuratezza del 90\%, allora giudichiamo che il classificatore A sia superiore.\\ 
Al contrario la \emph{recall} non è una singola metrica di valutazione numerica: due numeri per la valutazione del classificatore. Avere metriche di valutazione a più numeri rende più difficile confrontare gli algoritmi.
\\
Avere una singola metrica di valutazione del numero come l'\emph{accuratezza} consente di ordinare i modelli in base alle loro prestazioni su questa metrica e decidere rapidamente che cosa funziona meglio.

\subsection{Overfitting}
\label{subsec:overfitting}

L'\emph{overfitting} è "la produzione di un'analisi che corrisponde troppo o esattamente a un particolare insieme di dati, e può quindi non riuscire ad adattare dati aggiuntivi o prevedere in modo affidabile le osservazioni future".\\
La rete neurale deve avere capacità di comprensione del modello statistico dei dati, non memorizzare i soli dati del training set. L'essenza del sovraffollamento consiste nell'estrarre inconsapevolmente parte della variazione residua (cioè il rumore) come se quella variazione rappresentasse la sottostante struttura del modello \cite{burnham2003model}.\\
Per ridurre la possibilità di overfitting esistono diverse tecniche come la convalida incrociata (cross validation), la regolarizzazione o l'\emph{early stopping}, che consiste nell'utilizzo di un validation set di coppie non usate nel training set per la misurazione dell'errore.\\
La base di alcune tecniche è per penalizzare esplicitamente i modelli eccessivamente complessi per testare la capacità del modello di generalizzare, valutando le sue prestazioni su un insieme di dati non utilizzati per l'addestramento.

\section{Reti neurali per la classificazione}
\label{sec:classificazione}
La classificazione è la suddivisione di oggetti in insiemi disgiunti secondo un criterio stabilito a priori (etichetta). Ogni oggetto viene rappresentato come un vettore di numeri in modo da poter essere classificato dalla rete neurale, un vettore di feature che contraddistingue univocamente el'oggetto.\\
\subsection{Definizione}
\label{subsec:defclass}
Date $N$ classi di appartenenza tra cui discriminare, il vettore di input $x$ possiede $L$ dimensioni delle feature da classificare, il vettore $y$ individua la classe formata da $N$ valori. Il classificatore riceve in input il vettore $x$ e restituisce in uscita il vettore $y$ dove $y_i=1$ se l'oggetto con l'input $x$ appartiene alla classe $i$ e $y_j=0$ per $i\neq j$ per $i,j=1 \dots N$.\\
Questo corrisponde ad una mappatura tra i valori di input e di output, che può essere modellata come una funzione non lineare.

\section{Reti neurali convoluzionali}
\label{sec:cnn}
Una rete neurale convoluzionale (o CNN "Convolutional Neural Network") è una classe di reti artificiali avanzate e \emph{feed-forward}, composte da uno o più strati convoluzionali con livelli completamente connessi (corrispondenti a quelli tipici di ANN) \cite{kim2014convolutional}. Questa architettura consente ai CNN di sfruttare la struttura 2D dei dati di input.\\\\
Una convoluzione può essere considerata come una funzione a finestra scorrevole, detta \emph{kernel} o filtro, applicata a una matrice. 
Nell CNN usiamo le convoluzioni sul livello di input per calcolare l'output. Ciò si traduce in connessioni locali, in cui ogni regione dell'ingresso è collegata a un neurone nell'output. Ogni livello applica filtri diversi, in genere centinaia o migliaia, e combina i loro risultati. Durante la fase di addestramento, una CNN impara automaticamente i valori dei suoi filtri in base all'attività che si desidera eseguire. \\
Le CNN sono adatte per elaborare dati visivi e altri dati bidimensionali. Hanno mostrato risultati superiori in entrambe le applicazioni di immagine e parlato. Possono essere addestrati con back-propagation standard. Le CNN sono più facili da addestrare rispetto ad altre reti neurali regolari, profonde e feed-forward e hanno molti meno parametri da stimare.\\
Una CNN è composta da un input e un livello di output, oltre a più livelli nascosti. Gli strati nascosti di una CNN consistono tipicamente di strati convoluzionali, strati di raggruppamento, strati completamente connessi e livelli di normalizzazione.\\
Rappresentano una delle tecniche più popolari nel campo del riconoscimento delle immagini ma anche nell'elaborazione del linguaggio naturale (NLP "Natuaral Language Processing") \cite{manning1999foundations,FIXME}.\\
In quest'ultimo caso, l'input sono frasi o documenti rappresentati come una matrice. Ogni riga della matrice corrisponde a un token, in genere una parola. Tipicamente, questi vettori sono dei word embeddings (rappresentazioni a bassa dimensione) come word2vec \cite{mikolov2013distributed} o GloVe \cite{pennington2014glove}, ma potrebbero anche essere vettori unici che indicizzano la parola in un vocabolario. Nel NLP utilizziamo filtri che scorrono su righe complete della matrice (parole). Pertanto, la "larghezza" dei nostri filtri è solitamente uguale alla larghezza della matrice di input. L'altezza, o la dimensione della regione, può variare, ma le finestre tipicamente scorrono su 2-5 parole per volta. \\
Risulta che le CNN applicate ai problemi di NLP funzionino abbastanza bene, un esempio è il modello \emph{Bag of Words} che è stato l'approccio standard per anni e ha portato a risultati piuttosto buoni.

\chapter{Formulazione}
\label{chap:formulazione}

\section{Definizione del problema }
\label{sec:problem}

Partendo da materiale testuale presente in rete, in particolare un dataset messo a disposizione da \emph{Yelp Dataset Challenge} contenente \numprint{5200000} reviews relative a \numprint{174000} business di 11 aree metropolitane nel mondo, l'obiettivo è quello di estrarre da questi dati caratteristiche di personalità.\\ 
Il nostro scopo è di identificare un adeguato spazio semantico che permetta di definire la personalità dell'oggetto target a cui un determinato testo si riferisce.

\section{Descrizione del dataset}
\label{sec:dataset}

Come punto di partenza, viene messo a disposizione un dizionario di 637 aggettivi, che la letteratura psicologica definisce come marker dei cinque grandi tratti di personalità noti come \emph{Big Five}.
In particolare, questo vocabolario associa ad ogni aggettivo un vettore di cinque elementi in cui ogni elemento corrisponde al grado di presenza o assenza di una determinata caratteristica.
\begin{figure}[H]
	\centering
\begin{tabular}{lccccc}
	\toprule
	 \textbf{Adjective} \quad & \multicolumn{5}{c}{\textbf{OCEAN}} \\
	
\midrule
	Active  & 0,053194 & 0,237406 & 0,365915 & 0,116700 & -0,058669  \\
	Angry  & -0,004604 & -0,038453 & 0,020755 & -0,294754 & 0,590114 \\
	Boring & -0,069877 & -0,099754 & -0,478821 & -0,236462 & 0,118821\\
	\rule{7pt}{0\normalbaselineskip} \dots &   \dots 		&			 \dots &			\dots &			 \dots & \dots \\
	\bottomrule
\end{tabular}
\captionof{table}{Esempio di dizionario OCEAN}
\label{tab:ocean}
\end{figure}

\section{Calcolo della Ground Truth}
\label{sec:GroundTruth}

Per poter addestrare correttamente il modello è necessario avere una Ground Truth, ovvero una label associata ad ogni input della rete. Senza questa informazione sarebbe impossibile ottimizzare il modello e valutarne la validità. 

Si procede eliminando dal dataset tutte le sentences non contenenti almeno uno degli aggettivi presenti nel dizionario OCEAN. In seguito verranno utilizzati i dati contenuti all'interno del vocabolario per creare una mappatura diretta tra ogni frase e il corrispondente vettore di personalità, in cui ogni elemento sarà calcolato come la media del valore di ogni aggettivo presente nel testo.

\section{Preprocessing}
\label{sec:preprocessing}
Gli algoritmi di apprendimento automatico non sono in grado funzionare direttamente con il testo non elaborato, è quindi necessario eseguire in primis un preprocessamento del database di testi e in seguito convertire i dati in numeri, nello specifico, in vettori di numeri.\\
Alla fine di questo processo il dataset ottenuto sarà suddiviso in tre corpora: 
\begin{itemize}
	\item $70\%$ training set : contenente \numprint{4351900} reviews utilizzate dalla rete per l'apprendimento;
	\item $10\%$ validation set: contenente \numprint{621700} reviews;
	\item $20\%$ testing set: contenente \numprint{1243000} reviews.
\end{itemize}
La divisione dei tre dataset dovrà mantenere una buona distribuzione fra le diverse classi.

\subsection{Natural Language Processing}
\label{subsec:nlp}
Il \emph{Natural Language Processing} (NLP) è un insieme di tecniche di computer science e linguistica che ricorrono a dei calcolatori per analizzare il linguaggio umano.
\\

Dal punto di vista sintattico, al dataset viene applicata la seguente serie di operazioni:
\begin{itemize}
	\item \textbf{Rottura della frase}: dato un pezzo di testo vengono trovati i limiti della frase, spesso contrassegnati da punti o altri segni di punteggiatura.
	\item \textbf{Stemming}: alcune parole vengono ridotte alla loro forma radice (ad esempio ``argue, argued, argues, arguing, and argus'' sono mappati alla parola ``argu'').
	\item \textbf{Segmentazione di parole}: un blocco di testo o sentence viene separato in parole. Per una lingua come l'inglese, questo è abbastanza banale, poiché le parole sono solitamente separate da spazi. 
\end{itemize}
Dal punto di vista semantico invece si interviene nel seguente modo:
\begin{itemize}
	\item \textbf{Semantica lessicale}: tenta di comprendere il significato computazionale delle singole parole nel loro contesto.
	\item \textbf{Comprensione del linguaggio naturale}: i blocchi di testo vengono convertiti in rappresentazioni più formali e più facili da manipolare per i computer. 
\end{itemize}
Ricorrendo, dove possibile, alle tecniche sopraelencate, è possibile costruire un dizionario delle \numprint{60000} parole più frequenti nel corpus di training, basato sulla frequenza assoluta di una parola, avendo cura di eliminare tutti gli aggettivi presenti nel dataset OCEAN.

\begin{figure}[H]
	\centering
	{\includegraphics[width=.8\textwidth]{images/dict_histogram30}}
	\caption{Istogramma rappresentativo delle 30 parole più frequenti nel dizionario}
	\label{fig:Istrogramma del dizionario}
\end{figure}
In questo modo non verrà influenzata la rete mostrando l'associazione tra gli aggettivi e le label associate, ed il modello sarà ``costretto'' ad imparare il legame esistente tra il contesto di una frase e il relativo valore del tratto di personalità. 

Inoltre sarà necessario anche rimuovere tutte le \emph{stopwords} relative alla lingua inglese, ovvero parole considerate poco significative perché usate troppo frequentemente all'interno delle frasi --- per esempio gli articoli e le congiunzioni --- filtrando i termini comuni e senza uno specifico significato semantico dalle parole che trasportano vere informazioni.

In seguito ogni parola del dizionario verrà codificata con un valore intero univoco, mentre quelle non presenti, tra cui gli aggettivi mappati nel vocabolario OCEAN, verranno indicizzati al valore ``-1'' e gli verrà associato il token ``UNK'' come mostrato nella Figura~\ref{fig:preprocessing}
\begin{figure}[H]
	\centering
	{\includegraphics[width=.5\textwidth]{images/preprocessing}}
	\caption{Visualizzazione del preprocessing}
	\label{fig:preprocessing}
\end{figure}


\chapter{Esperimenti e risultati}
\label{chap:esperimenti}

La maggior parte degli attuali studi di previsione della personalità si sono concentrati sull'applicazione di tecniche generali di apprendimento automatico per predire i tratti di personalità Big Five.
In particolare verranno utilizzate diverse strutture di rete, combinando differenti domini.

\section{Spazi di rappresentazione}
\label{sec:approcci}
Per affrontare questo problema di \emph{Text Mining}, gli esperimenti si concentrano su due principali metodi per l'estrazione delle caratteristiche del testo:
\begin{itemize}
	\item Un approccio supervisionato, in cui viene utilizzato come strumento di generazione di feature un vettore \emph{bag-of-words} o brevemente BoW, una rappresentazione di testo che descrive la presenza delle parole all'interno di un documento, in questo caso nel dizionario delle occorrenze \cite{wallach2006topic}.
	\item Un approccio non supervisionato, in cui viene costruito un embedding, tramite l'algoritmo \texttt{word2vec} di Tomas Mikolov \cite{mikolov2013distributed}. Insegnando alla rete il significato delle parole e la relazione tra di esse, è possibile rappresentare, sotto forma di vettori, le mappature tra le parole e i contesti.
\end{itemize}
$\\$
In seguito viene posta l'attenzione su tre diverse architetture neurali:
\begin{itemize}
	\item Reti fully-connected;
	\item Reti neurali convoluzionali CNN;
	\item Classificatori multi-label binari.
\end{itemize}



\section{Esperimento 1}
\label{sec:es1}
\subsection{Input Features}
\label{subsec:features1}

Nel primo approccio proposto, per rappresentare i dati testuali viene utilizzato il modello \emph{bag-of-words}, un tipo di descrizione semplificata, spesso utilizzata nell'elaborazione del linguaggio naturale e nel campo dell'\emph{Information Retrivial} (IR). 

Sfruttando il dizionario delle occorrenze precedentemente costruito, il testo viene modellato  come fosse una ``borsa di parole'', in cui grammatica e ordine delle parole vengono trascurate.
Ogni frase viene ridotta ad un vettore in cui ogni elemento identifica una parola del dizionario. Nella posizione corrispondente ad un determinato termine vi sarà il valore \num{1} se la parola è contenuta nella frase, \num{0} altrimenti.

Il modello riguarda solo le parole conosciute, di conseguenza i vocaboli che compaiono nel testo ma sono assenti nel dizionario vengono trascurati.

\begin{figure}[H]
	\centering
	{\includegraphics[width=.6\textwidth]{images/bow}}
	\caption{Visualizzazione del modello \emph{bag-of-words}}
	\label{fig:bow}
\end{figure}

\subsection{Architettura della rete}
\label{subsec:modelli1}

Una volta estratte, le features posso essere passate in input alla rete neurale, che le elabora calcolando le risposte dei neuroni dal livello di input verso il livello di output.

In questa prima strategia viene utilizzata una rete \emph{feed-forward} con struttura densa o \emph{fully-connected}. Per i dettagli di queste architetture si rimanda alla sezione \ref{subsec:fc}.

Ad ogni livello viene applicata la funzione di attivazione non lineare \emph{ReLU}, descritta nella sezione \ref{subsec:fattivazione}.

Inoltre, per accelerare l'apprendimento ed aumentare la stabilità della rete, viene effettuata dopo ogni layer una \emph{Batch Normalization}, definita nella sezione \ref{subsec:normalization}. 

Vengono presentate nella tabella \ref{tab:arcbow+fc} tre architetture implementate, con differenti strati e numero di neuroni che caratterizza ciascuno.


\begin{figure}[H]
	\centering
		\begin{tabular}{lcccc}
			\toprule
			\textbf{Layer} \quad & \textbf{Modello 1} & \textbf{Modello 2} & \textbf{Modello 3} \\
			\midrule
			Input 				 & \numprint{60000}	  & \numprint{60000}   &\numprint{60000}\\
			fc1  				 & \num{300}		  & \num{300} 		   & \num{100} 		\\
			fc2  				 &  \num{200}		  & \num{200} 		   & \num{50}  		\\
			fc3					 & -				  & \num{100} 		   & \num{20}    	\\
			Output 				 &  \num{5}			  & \num{5} 		   & \num{5}		\\
			\bottomrule
		\end{tabular}
	\captionof{table}{Confronto delle architetture di tre differenti modelli}
	\label{tab:arcbow+fc}
\end{figure}


Tutte le simulazioni sono state addestrate per \num{10} epoche: la fase di addestramento viene in genere portata avanti fino a quando le performance sul test non producono alcun miglioramento.

L’ottimizzatore scelto è \emph{Adagrad}, introdotto nella sezione \ref{subsubsec:adagrad}, con learning rate \numprint{0,001}.

Come funzione di loss è stato scelto l'errore quadratico medio, in inglese \emph{Mean Squared Error (MSE)}, presentato nella sezione \ref{subsubsec:MSE}. Mentre la metrica di valutazione utilizzata per misurare le prestazioni predittive del modello è la \emph{Root Mean Squared Error} (RMSE), introdotta nella sezione \ref{subsubsec:regressione}.

In ogni epoca si alternano una fase di training ed una fase di test in modo tale da monitorare costantemente i miglioramenti o i peggioramenti del modello sul test set. 

\subsection{Performance}
\label{subsec:performance1}
Prendendo in considerazione le tre diverse architetture implementate, vengono presentati i risultati ottenuti in termini di loss.
\begin{table}[H]
	\centering
	\begin{tabular}{l@{\hspace{.5cm}}ccc}
		\toprule
		 & \textbf{Train loss} & \textbf{Test loss} & \textbf{Tempo di training}  \\
		\midrule
		\textbf{Modello 1} & \numprint{0.061} & \numprint{0.062} &\numprint{235} min \\
		\textbf{Modello 2} & \numprint{0.090} & \numprint{0.061} &\numprint{250} min \\
		\textbf{Modello 3} & \numprint{0.068} & \numprint{0.062} &\numprint{265} min \\
%		\textbf{Modello 4} & \numprint{0.0287} & \numprint{0.0104} &\numprint{145} min \\
		\bottomrule 
	\end{tabular}
	\captionof{table}{Confronto dei risultati in termini di {loss} ottenuti dalle tre diverse reti}
	\label{tab:lossbow+fc}
\end{table}

Per valutare l'efficacia di questi modelli, è fondamentale eseguire un analisi dettagliata, in particolare ponendo l'attenzione sui valori di RMSE per ogni tratto di personalità.
Calcolando il valore medio assunto da ogni caratteristica durante la fase di addestramento, è possibile stabilire qual è il valore di Root Mean Squared Error di un modello concettuale, chiamato ``Modello 0'', che per ogni tratto predice sempre il suo valore medio.
 
\begin{figure}[H]
	\centering
	\begin{tabular}{clccccc}
		\toprule	
		& 		  & \multicolumn{5}{c}{\textbf{Root Mean Squared Error}} 									    \\
		\multicolumn{2}{c}{\multirow{-2}{*}{Modelli}}
		& O 				& C 			   & E 				  & A 				 & N 			    \\ 
		\midrule
		\multirow{2}*{\textbf{Modello 1}} & Modello   & \numprint{0,148} & \numprint{0,227} & \numprint{0,224} & \numprint{0,251} & \numprint{0,351} \\
		& Modello 0 & \numprint{0,145} & \numprint{0,224} & \numprint{0,213} & \numprint{0,218} & \numprint{0,318} \\
		\midrule
		\multirow{2}*{\textbf{Modello 2}} & Modello   & \numprint{0,147} & \numprint{0,226} & \numprint{0,225} & \numprint{0,251} & \numprint{0,341} \\
		& Modello 0 & \numprint{0,141} & \numprint{0,227} & \numprint{0,213} & \numprint{0,208} & \numprint{0,305} \\
		\midrule
		\multirow{2}*{\textbf{Modello 3}} & Modello   & \numprint{0,147} & \numprint{0,226} & \numprint{0,225} & \numprint{0,262} & \numprint{0,348} \\
		& Modello 0 & \numprint{0,233} & \numprint{0,307} & \numprint{0,262} & \numprint{0,373} & \numprint{0,546}  \\
%		\midrule
%		\multirow{2}*{\textbf{Modello 4}} & Modello   & \numprint{0,147} & \numprint{0,22 s6} & \numprint{0,225} & \numprint{0,262} & \numprint{0,348} \\
%		& Modello 0 & \numprint{0,233} & \numprint{0,307} & \numprint{0,262} & \numprint{0,373} & \numprint{0,546}  \\
		\bottomrule	
	\end{tabular}
	\captionof{table}{Confronto dei risultati in termini di {Root Mean Squared Error} delle architetture contro il modello basato sulla media di training}
	\label{tab:confmm0bow+fc}
\end{figure}

\begin{figure}[htb]
	\centering
	{\includegraphics[width=.75\textwidth]{images/loss1}} 
	\caption{Visualizzazione delle training loss dei tre modelli}
	\label{fig:loss}
\end{figure}


Mettendo a confronto le due metriche, si nota che i modelli realmente implementati imparano nella maggioranza dei casi a predire un valore con lo stesso errore commesso dai modelli banali che calcolano la media. Risulta allora evidente che i risultati ottenuti non siano ottimali.

Nonostante ciò, il terzo modello sembrerebbe essere leggermente migliore degli altri due, in particolare confrontandolo con il corrispondente modello nullo si nota la presenza di un piccolo margine di miglioramento.

La scarsa efficienza di questi modelli potrebbe dipendere dall'efficacia con cui vengono codificate le feature in input alla rete. 
Infatti le limitazioni dell'approccio bag-of-words derivano in parte dalla progettazione del vocabolario e della sua dimensione, che può causare una scarsa ``descrizione'' del documento. 
Scartare l'ordine delle parole e ignorare il contesto non consente di determinare la differenza tra le stesse parole disposte diversamente, i sinonimi ecc.
Inoltre, un tipo di rappresentazione sparsa risulta più difficile da modellare quando si cercano modelli in grado di sfruttare poche informazioni in uno spazio rappresentativo ampio, sia per ragioni computazionali (spazio e complessità temporale) sia per ragioni di informazione.

\section{Esperimento 2}
\label{sec:es2}

Durante l'applicazione di tecniche di apprendimento automatico, tutte le ``informazioni'' vengono rappresentate per mezzo di identificativi unici e discreti. 

Nel caso dell'approccio {BoW}, la codifica utilizzata non fornisce alcuna informazione utile al sistema riguardo le relazioni che possono sussistere tra i singoli elementi. Ciò significa che quando sta elaborando i dati, il modello può sfruttare molto poco di ciò che ha appreso su un determinato termine. 

Inoltre la ``raffigurazione'' utilizzata nel precedente esperimento, ha portato alla creazione di dati sparsi. Di conseguenza, per ottenere un modello di successo, un'alternativa valida sarebbe quella di sfruttare {modelli spaziali vettoriali}, in inglese \emph{Vector Space Model} (VSM), per rappresentare le parole in uno spazio continuo \cite{mikolov2013linguistic,erk2008structured}.
Questi metodi dipendono dall'ipotesi distributiva, la quale afferma che le parole che appaiono negli stessi contesti condividono lo stesso significato semantico \cite{baroni2014don}. 

\subsection{Input Features}
\label{subsec:features2}

Nel secondo approccio, viene applicato l'algoritmo non supervisionato \texttt{word2vec} di Tomas Mikolov  \cite{mikolov2013efficient}. 
\texttt{Word2vec} è un modello predittivo particolarmente efficiente dal punto di vista computazionale per l'apprendimento degli embedding di parole a partire dal testo non elaborato.
Esso è basato su una rete neurale artificiale a due strati, addestrati a ricostruire i contesti linguistici delle parole. 

A partire dal corpus di testo, la rete prende in input un set formato dall'accoppiamento di ogni parola target e i contesti in cui appare e restituisce un insieme di vettori che rappresentano la distribuzione semantica delle parole nel testo. 

Viene considerato come ``contesto'' l'insieme delle ``parole a sinistra'' e delle ``parole alla destra'' dell'obiettivo, ovvero la finestra di dimensione 1 attorno all'elemento target. Ogni coppia di destinazione del contesto viene trattata come se fosse una nuova osservazione, incrementando le informazioni distribuzionali. Viene così prodotto uno spazio vettoriale di diverse centinaia di dimensioni, in cui ogni parola univoca viene assegnata a un vettore corrispondente nello spazio.

Per l'implementazione viene utilizzato il modello \emph{skip-gram}, una versione di \texttt{word2vec} che vuole predire le parole del contesto di origine (label) a partire dalle parole target (features).

\begin{figure}[H]
	\centering
	{\includegraphics[width=.7\textwidth]{images/skip-gram}} 
	\caption{Visualizzazione del modello skip-gram}
	\label{fig:mikolov}
\end{figure}

In questo tipo di apprendimento, i vettori si posizionano nello spazio in modo tale che le parole che condividono contesti comuni nel corpo siano situate in stretta prossimità l'una dell'altra. Un esempio interessante viene illustrato nella figura \ref{fig:embedding1}\subref{subfig:vis3d} --- si ponga particolare attenzione alle parole ``beautiful'', ``lovely'', ``chic'', ``trendy'' ecc.. --- in cui i vocaboli semanticamente simili si trovano vicini dello spazio.

\begin{figure}[H]
	\centering
	\subfloat[][\emph{Visualizzazione 2D}\label{subfig:vis2d}]
	{\includegraphics[width=.4\textwidth]{images/embedding1/embedding1_tsne_2d}}
	\hspace{10mm}
	\subfloat[][\emph{Dettaglio di una visualizzazione 3D}\label{subfig:vis3d}]
	{\includegraphics[width=.45\textwidth]{images/embedding1/Embedding1_similarity}}
	
	\caption{Proiezione dell'embedding di Mikolov nello spazio 2-3 dimensionale, tramite la \emph{tecnica di riduzione della dimensionalità} (t-SNE) \cite{maaten2008visualizing}}
	\label{fig:embedding1}
\end{figure}

La funzione obiettivo utilizzata dalla rete per la costruzione dell'embedding viene definita sull'intero set di dati, ed ottimizzata con la \emph{Stochastic Gradient Descent} (SGD), definita nella sezione \ref{subsubsec:SGD}.

Nella seguente tabella vengono presentati i due embedding realizzati e i relativi parametri.

\begin{figure}[htb]
	\centering
	\begin{tabular}{ccc}
		\toprule	
		 		  				& \multicolumn{2}{c}{\textbf{Parameters}}	\\
		{\multirow{-2}{*}{Embedding}}
								& Embedding Size 	& Num Sampled 	 		\\ 
		\midrule
		\textbf{Embedding 1}    & \numprint{40} 	& \numprint{20}  		\\
		\midrule
		\textbf{Embedding 2}    & \numprint{250} 	& \numprint{50}  		\\
		\bottomrule	
	\end{tabular}
	\captionof{table}{Confronto dei parametri della rete impostati per la realizzazione dell'embedding}
	\label{tab:confemb}
\end{figure}

\subsection{Architettura della rete}
\label{subsec:modelli2}

La rappresentazione spazio-vettoriale viene utilizzata come feature della modello che si andrà a costruire.
In questo secondo approccio verrà utilizzata un \emph{rete convoluzionale}, in cui ogni neurone è collegato solo a pochi neuroni vicini nel livello precedente, e lo stesso insieme di pesi viene utilizzato per ogni neurone.\\

In questo tipo di rete, il modello di connessione locale e lo schema di peso condiviso possono essere interpretati come un filtro (o un insieme di filtri) che accettano un sottoinsieme dei dati di input alla volta, ma vengono applicati all'intero input.

Uno strato convoluzionale è molto più specializzato ed efficiente di uno completamente connesso.

\begin{figure}[H]
	\centering
	{\includegraphics[width=.85\textwidth]{images/cnn}} 
	\caption{Esempio di architettura convoluzionale per la manipolazione del linguaggio naturale}
	\label{fig:rmse}
\end{figure}

Le operazioni eseguite da questi strati vengono trasformate in ``moltiplicazioni'' non lineari tramite l'applicazione della funzione di attivazione \emph{ReLU}, introdotta nella sezione \ref{subsec:fattivazione}.

Ad ogni livello convoluzionale viene applicata una \emph{Batch Normalization}, definita nella sezione \ref{subsec:normalization}, utilizzata sull'input per il ridimensionamento delle funzionalità e la normalizzazione batch nei livelli nascosti.

Dopo il primo strato convoluzionale viene inserito un \emph{layer di pooling}, introdotto nella sezione \ref{subsec:maxpool}, necessario per ridurre in modo efficace i campioni dell'output del livello precedente, riducendo il numero di operazioni richieste per tutti i livelli successivi, ma passando comunque le informazioni valide.

Come ultimo strato della rete viene scelto un \emph{layer fully-connected}, definito nella sezione \ref{subsec:fc}, corrispondente ad un'operazione lineare sul vettore di input del livello che esegue una serie di trasformazioni sulla rappresentazione profonda al fine di emettere i punteggi di ogni classe.

\begin{figure}[H]
	\centering
	\begin{tabular}{lccc}
		\toprule
		\textbf{Layer}& \textbf{Modello 4} & \textbf{Modello 5} & \textbf{Modello 6} 		\\ 
		\midrule
		conv1 	& \num{10}$\times$\num{5}, 10	  & \num{5}$\times$\num{5}, 150, same pad    &\num{3}$\times$\num{3}, 100, same pad 		   \\
		
		mpool1 	& &{\num{4}$\times$\num{4}, stride 2, same pad}	&   \\
		conv2  	& \num{18}$\times$\num{18}, 10	  &  \num{5}$\times$\num{20}, 100, same pad	  &		\num{3}$\times$\num{20}, 75, same pad    \\
		conv3  	& ---	  & \num{1}$\times$\num{20}, 50, 	   &	\num{1}$\times$\num{20}, 50 	   \\
		fcout		& &{\num{1}$\times$\num{1}, 5}&		   \\
		
		\bottomrule	
	\end{tabular}
	\captionof{table}{Architetture implementate a partire dall'embedding 1}
	\label{tab:netemb1}
\end{figure}

\begin{figure}[H]
	\centering
	\begin{tabular}{lcc}
		\toprule
		\textbf{Layer}& \textbf{Modello 7} 								  & \textbf{Modello 8} 			   \\ 
		\midrule
		conv1 	& \num{3}$\times$\num{3}, 100, stride 2, same pad     & ---	   \\
		mpool1 	& \num{4}$\times$\num{4}, stride 2, same pad		  & ---	   \\
		conv2  	& \num{3}$\times$\num{63}, 75, stride 2, same pad	  & ---    \\
		conv3  	& \num{1}$\times$\num{32}, 50, stride 2	  				  & ---	   \\
		fc1  	& ---													  & \num{1}$\times$\num{1}, 100	   \\
		fc2  	& \num{1}$\times$\num{32}, 50, stride 2	  				  & \num{1}$\times$\num{1},  50    \\
		fc3  	& \num{1}$\times$\num{32}, 50, stride 2	  				  & \num{1}$\times$\num{1},  20	   \\
		fcout	& \num{1}$\times$\num{1}, 5   			  				  & \num{1}$\times$\num{1},   5	   \\
		\bottomrule	
	\end{tabular}
	\captionof{table}{Architetture implementate a partire dall'embedding 2}
	\label{tab:netemb2}
\end{figure}

Vengono presentate nelle tabelle \ref{tab:netemb1} e \ref{tab:netemb2} le diverse configurazioni dei layer convoluzionali delle architetture implementate per i due embedding.\\

L'input della rete è costituito da un numero di caratteristiche pari a $n \times p$ dove $n$ è la dimensione dell'embedding e $p$ il numero delle parole di ogni sentence.

Come algoritmo di ottimizzazione viene scelto sempre \emph{Adagrad}, illustrato nella sezione \ref{subsubsec:adagrad}, mentre il valore di \emph{learning rate} settato per ogni modello viene mostrato nella tabella \ref{tab:learningratemikolov}.

Viene mantenuta anche in questa simulazione la funzione di costo MSE, approfondita nella sezione \ref{subsubsec:MSE}.

Nel caso del secondo embedding si è voluto provare anche un modello che sfruttasse solo livelli di rete \emph{fully-connected}, senza applicare alcuna convoluzione. 


\begin{table}[t]
	\centering
	\begin{tabular}{llc}
		\toprule
		\multicolumn{2}{c}{{Modelli}} & \textbf{Learning Rate}  \\
		\midrule
		\multirow{3}*{{Embedding 1}} 
		&\textbf{Modello 4} & \numprint{0.0010} \\
		&\textbf{Modello 5} & \numprint{0.0001} \\
		&\textbf{Modello 6} & \numprint{0.0050} \\
		\midrule
		\multirow{2}*{{Embedding 2}} 
		&\textbf{Modello 7} & \numprint{0.0050} \\
		&\textbf{Modello 8} & \numprint{0.0001} \\	
		\bottomrule 
	\end{tabular}
	\captionof{table}{Learning rate delle simulazioni effettuate nell'esperimento 2}
	\label{tab:learningratemikolov}
\end{table}

\subsection{Performance}
\label{subsec:performance2}

Prendendo in considerazione i due diversi embedding, vengono messe a confronto a confronto le diverse architetture implementate, mostrando i risultati ottenuti in termini di loss.
\begin{table}[H]
	\centering
	\begin{tabular}{ll@{\hspace{.5cm}}ccc}
		\toprule
		\multicolumn{2}{c}{{Modelli}} & \textbf{Train loss} & \textbf{Test loss} & \textbf{Tempo di training}  \\
		\midrule
		\multirow{3}*{{Embedding 1}} 
		&\textbf{Modello 4} & \numprint{0.061} & \numprint{0.058} &\numprint{200} min \\
		&\textbf{Modello 5} & \numprint{0.052} & \numprint{0.060} &\numprint{310} min \\
		&\textbf{Modello 6} & \numprint{0.042} & \numprint{0.060} &\numprint{540} min \\
		\midrule
		\multirow{2}*{{Embedding 2}} 
		&\textbf{Modello 7} & \numprint{0.038} & \numprint{0.057} &\numprint{225} min \\
		&\textbf{Modello 8} & \numprint{0.058} & \numprint{0.117} &\numprint{250} min \\	
		\bottomrule 
	\end{tabular}
	\captionof{table}{Confronto dei risultati in termini di \emph{loss} ottenuti nelle diverse reti con i due diversi embedding}
	\label{tab:lossmikolov}
\end{table}

Valutando le prestazioni su train e test set viene rilevato il ``Modello 7'' come il migliore tra quelli proposti, presentando i valori di loss più bassi. 
Oltre a ciò, anche rispetto al precedente approccio il ``Modello 7'' mostra una performance migliorata circa 
del \SI{3}{\percent}.

Vengono analizzati anche i valori di RMSE, relativi ad ogni tratto di personalità, riassunti nella tabella \ref{tab:rmsemikolov}, e viene inoltre visualizzata la curva RMSE nella figura \ref{fig:rmse}.

\begin{figure}[H]
	\centering
	{\includegraphics[width=.65\textwidth]{images/rmse2-linlong}} 
	\caption{Visualizzazione delle training RMSE dei modelli}
	\label{fig:rmse}
\end{figure}

\begin{figure}[t]
	\centering
	\begin{tabular}{clccccc}
		\toprule	
		& 		 			& \multicolumn{5}{c}{\textbf{Root Mean Squared Error}} 									       \\
		\multicolumn{2}{c}{\multirow{-2}{*}{Modelli}}
		& O 				& C 			   & E 				  & A 				 & N 			   \\ 
		\midrule
		\multirow{3}*{{Embedding 1}} 
		& \textbf{Modello 4} & \numprint{0,148} & \numprint{0,226} & \numprint{0,232} & \numprint{0,252} & \numprint{0,313} \\
		& \textbf{Modello 5} & \numprint{0,146} & \numprint{0,227} & \numprint{0,230} & \numprint{0,251} & \numprint{0,336} \\
		& \textbf{Modello 6} & \numprint{0,146} & \numprint{0,225} & \numprint{0,224} & \numprint{0,251} & \numprint{0,337} \\
		\midrule
		\multirow{2}*{{Embedding 2}} 
		& \textbf{Modello 7} & \numprint{0,147} & \numprint{0,223} & \numprint{0,222} & \numprint{0,251} & \numprint{0,320} \\
		& \textbf{Modello 8} & \numprint{0,399} & \numprint{0,275} & \numprint{0,257} & \numprint{0,271} & \numprint{0,457} \\
		\bottomrule	
	\end{tabular}
	\captionof{table}{Confronto dei risultati in termini di {Root Mean Squared Error} delle architetture sul test set}
	\label{tab:rmsemikolov}
\end{figure}


\section{Esperimento 3}
\label{sec:es3}

Nel terzo approccio si decide di valutare una rappresentazione dell'input alternativa.

\subsection{Input Features}
\label{subsec:features3}

Ricorrendo nuovamente al modello \emph{skip-gram}, l'input della rete questa volta comprende il set formato dall'accoppiamento tra gli aggettivi contenuti nel dizionario OCEAN e i loro contesti.

La finestratura che si andrà a definire in torno al target sarà di dimensione 2 e considererà le due parole a sinistra e le due parole a destra dell'aggettivo. La window non sarà più quindi incentrata su ogni elemento della sentence.
Dunque nell'embedding che si andrà a costruire non verranno appresi i contesti di tutte le parole, ma solamente quelli di nostro interesse, ovvero relativi agli aggettivi.

In questo esperimento viene realizzato un solo embedding, la cui dimensione è \num{250} e con numero di etichette negative da campionare pari a 50 \cite{liu2016classification}.

\subsection{Architettura della rete}
\label{subsec:modelli3}

Viene utilizzato l'embedding estratto per costruire due modelli di reti neurali, le cui architetture vengono presentate nella tabella \ref{tab:netemb3}.
Come nel precedente esperimento si ricorrerà alle \emph{reti convoluzionali}; si rimanda alla sezione \ref{subsec:modelli2} per i dettagli. 

\begin{figure}[H]
	\centering
	\begin{tabular}{lcc}
		\toprule
		\textbf{Layer}& \textbf{Modello 9} & \textbf{Modello 10}	\\ 
		\midrule
		conv1   & \num{7}$\times$\num{5}, 100, stride 2, same pad    		&\num{3}$\times$\num{3}, 100, stride 2, same pad 		   \\
		mpool1 	&\multicolumn{2}{c}{\num{4}$\times$\num{4}, stride 2, same pad}	 \\
		conv2  	&  \num{5}$\times$\num{63}, 75, stride 2, same pad	    &		\num{3}$\times$\num{63}, 75, stride 2, same pad    \\
		conv3  	& \num{3}$\times$\num{32}, 50, stride 2, same pad 	   	&	\num{1}$\times$\num{32}, 50, stride 2 	   \\
		conv4  	& \num{1}$\times$\num{16}, 25, stride 2 	   	&	--- 	   \\
		fcout	&\multicolumn{2}{c}{\num{1}$\times$\num{1}, 5}						\\
		
		\bottomrule	
	\end{tabular}
	\captionof{table}{Architetture dei due modelli implementati a partire dall'embedding 3}
	\label{tab:netemb3}
\end{figure}

L'algoritmo di apprendimento utilizzato è \emph{Adagrad}, definito nella sezione \ref{subsubsec:adagrad}, con \emph{learning rate} \numprint{0,0005} per il ``Modello 9'' e \numprint{0,005} per il ``Modello 10''. La funzione di \emph{loss} scelta è la MSE, introdotta nella sezione \ref{subsubsec:MSE}. 

\subsection{Performance}
\label{subsec:performance3}

Vengono presentati nella tabella \ref{tab:lossmikolov2} i valori ottenuti dalla funzione obiettivo dei due modelli, mentre nella tabella \ref{tab:rmsemikolov2} vengono riassunti i valori di RMSE per ogni tratto di personalità.

\begin{table}[H]
	\centering
	\begin{tabular}{l@{\hspace{.5cm}}ccc}
		\toprule
		& \textbf{Train loss} & \textbf{Test loss} & \textbf{Tempo di training}  \\
		\midrule
		\textbf{Modello 9} & \numprint{0.043} & \numprint{0.060} &\numprint{18} h \\
		\textbf{Modello 10} & \numprint{0.050} & \numprint{0.059} &\numprint{17} h \\	
		\bottomrule 
	\end{tabular}
	\captionof{table}{Confronto dei risultati in termini di {loss} ottenuti nell'implementazione dei due modelli}
	\label{tab:lossmikolov2}
\end{table}

\begin{figure}[H]
	\centering
	\begin{tabular}{clccccc}
		\toprule	
		& 		 			& \multicolumn{5}{c}{\textbf{Root Mean Squared Error}} 									       \\
		\multicolumn{2}{c}{\multirow{-2}{*}{Modelli}}
		& O 				& C 			   & E 				  & A 				 & N 			   \\ 
		\midrule
		& \textbf{Modello 9} & \numprint{0,146} & \numprint{0,223} & \numprint{0,223} & \numprint{0,251} & \numprint{0,331} \\
		& \textbf{Modello 10} & \numprint{0,147} & \numprint{0,225} & \numprint{0,224} & \numprint{0,252} & \numprint{0,339} \\
		\bottomrule	
	\end{tabular}
	\captionof{table}{Confronto dei risultati in termini di {RMSE} delle due architetture sul test set}
	\label{tab:rmsemikolov2}
\end{figure}

Rispetto al precedente approccio non vengono ottenuti dei miglioramenti. 

\section{Esperimento 4}
\label{sec:es4}

L'ultimo approccio provato riutilizza i precedenti metodi di estrazione di feature che hanno ottenuto le migliori prestazioni per realizzare un modello predittivo di classificazione.

Nello specifico il problema viene trasformato in un compito di classificazione binaria multi-label, in cui si cercano di predire cinque diverse etichette per ogni istanza. 

Per ogni dimensione di personalità l'output della rete sarà ``0'', se il valore reale che assume un determinato tratto è inferiore a 0, o ``1'' altrimenti.
\\
Viene stabilito 0 come ``punto di neutralità assoluta'' perché corrisponde alla media dei valori osservati in ognuna delle caratteristiche.

Adottando questo metodo, si assumerà di poter estrarre da questa funzione una misura di polarità, positiva o negativa, che indicherà un tratto più o meno accentuato.

Il vantaggio effettivo di questo approccio consiste nella possibilità di valutare le performance in modo standard, visualizzando le matrici di confusione e misurandone l'accuratezza, argomenti trattati nella sezione \ref{subsubsec:classificazione}.

\subsection{Input Features}
\label{subsec:features4}

Vengono utilizzati come ingresso della rete due embedding di Mikolov realizzati negli esperimenti precedenti, ovvero l'``Embedding 2'' e l'``Embedding 3''.

\subsection{Architettura della rete}
\label{subsec:modelli4}

Le reti che vengono realizzate sono molto simili a quelle precedentemente implementate; i dettagli delle architetture vengono presentati nella tabella \ref{tab:netemb4}.

Ad ogni livello della rete viene applicata la stessa procedura degli esperimenti precedenti. 

\begin{figure}[H]
	\centering
	\begin{tabular}{lccc}
		\toprule
		\textbf{Layer} & \textbf{Modello 11} & \textbf{Modello 12} & \textbf{Modello 13}  	\\ 
		\midrule
		conv1 	& \num{3}$\times$\num{3}, 100	  & \num{5}$\times$\num{3}, 100 & \num{7}$\times$\num{5}, 100 	   \\
		mpool1 	& {\num{4}$\times$\num{4}}	&{\num{4}$\times$\num{4}}	& {\num{4}$\times$\num{4}}	 \\
		conv2  	& \num{3}$\times$\num{63}, 75 &  \num{3}$\times$\num{63}, 75	  &		\num{5}$\times$\num{63}, 75    \\
		conv3  	&\num{1}$\times$\num{32}, 50	  & \num{1}$\times$\num{32}, 50  &	\num{3}$\times$\num{32}, 50 	  \\
		conv4  	& ---	  & \num{1}$\times$\num{16}, 25  &	\num{1}$\times$\num{16}, 25 	  \\
		fcout	&{\num{1}$\times$\num{1}, 10} &{\num{1}$\times$\num{1}, 10}&{\num{1}$\times$\num{1}, 10}		   \\
		\bottomrule	
	\end{tabular}
	\captionof{table}{Architetture implementate a partire dall'embedding 2}
	\label{tab:netemb4}
\end{figure}

In ogni livello convoluzionale della rete si utilizzano uno stride di dimensione 2 e padding same, tranne che per l'ultimo livello convoluzionale con padding valid, per i dettagli si consiglia di visitare la sezione { \ref{subsec:cnn}}.

\begin{figure}[H]
	\centering
	\begin{tabular}{lcc}
		\toprule
		\textbf{Layer} & \textbf{Modello 14} & \textbf{Modello 15}   	\\ 
		\midrule
		conv1 	& {\num{7}$\times$\num{5}, 100, stride 2, same pad}&{\num{7}$\times$\num{5}, 100, stride 2, same pad} \\
		mpool1 	& {\num{4}$\times$\num{4}, stride 2, same pad} &{\num{4}$\times$\num{4}, stride 2, same pad}  \\
		conv2  	& {\num{5}$\times$\num{63}, 75, stride 2, same pad}&{\num{5}$\times$\num{63}, 75, stride 2, same pad}    \\
		conv3  	&  {\num{3}$\times$\num{32}, 50, stride 2, same pad} &{\num{3}$\times$\num{32}, 50, stride 2, same pad} 	  \\
		conv4  	& \num{1}$\times$\num{16}, 25, stride 2  &	\num{3}$\times$\num{16}, 25, stride 2, same pad 	  \\
		conv5  	& ---  &	\num{1}$\times$\num{8}, 16, stride 2 	  \\
		fcout	& \multicolumn{2}{c}{\num{1}$\times$\num{1}, 10}		   \\
		\bottomrule	
	\end{tabular}
	\captionof{table}{Architetture implementate a partire dall'embedding 3}
	\label{tab:rmsebin2}
\end{figure}

In tutte le simulazioni l’ottimizzatore scelto è \emph{Adagrad}, introdotto nella sezione \ref{subsubsec:adagrad}, il valore di \emph{learning rate}  di ogni modello viene mostrato nella tabella \ref{tab:learningratemikolov2}.

Come funzione obiettivo viene utilizzata la \emph{Softamax Cross Entropy}, presentata nella sezione \ref{subsubsec:sce}. 

\begin{table}[H]
	\centering
	\begin{tabular}{llc}
		\toprule
		\multicolumn{2}{c}{{Modelli}} & \textbf{Learning Rate}  \\
		\midrule
		\multirow{3}*{{Embedding 2}} 
		&\textbf{Modello 11} & \numprint{0.0001} \\
		&\textbf{Modello 12} & \numprint{0.0005} \\
		&\textbf{Modello 13} & \numprint{0.0005} \\
		\midrule
		\multirow{2}*{{Embedding 3}} 
		&\textbf{Modello 14} & \numprint{0.0005} \\
		&\textbf{Modello 15} & \numprint{0.0005} \\	
		\bottomrule 
	\end{tabular}
	\captionof{table}{Learning rate delle simulazioni effettuate nell'esperimento 2}
	\label{tab:learningratemikolov2}
\end{table}


\subsection{Performance}
\label{subsec:performance4}

Prendendo in considerazione le diverse architetture implementate, vengono presentati i risultati ottenuti in termini di loss.

\begin{table}[H]
	\centering
	\begin{tabular}{ll@{\hspace{.5cm}}ccc}
		\toprule
		\multicolumn{2}{c}{{Modelli}} & \textbf{Train loss} & \textbf{Test loss} & \textbf{Tempo di training}  \\
		\midrule
		\multirow{3}*{{Embedding 2}} 
		&\textbf{Modello 11} & \numprint{3.275} & \numprint{3.455} &\numprint{20} h \\
		&\textbf{Modello 12} & \numprint{3.297} & \numprint{3.459} &\numprint{30} h \\
		&\textbf{Modello 13} & \numprint{3.444} & \numprint{3.458} &\numprint{40} h \\
		\midrule
		\multirow{2}*{{Embedding 2}} 
		&\textbf{Modello 14} & \numprint{3.350} & \numprint{3.459} &\numprint{20} h \\
		&\textbf{Modello 15} & \numprint{3.230} & \numprint{3.454} &\numprint{60} h \\	
		\bottomrule 
	\end{tabular}
	\captionof{table}{Confronto delle \emph{loss} ottenute nelle diverse reti con i due diversi embedding}
	\label{tab:lossmikolov3}
\end{table}

Come già detto, l'utilizzo di questo approccio consente una valutazione non più solo in termini di loss ma se ne potrà analizzare anche l'accuratezza, introdotta nella sezione \ref{subsubsec:classificazione}. 

\begin{figure}[H]
	\centering
	\begin{tabular}{clP{1cm}P{1cm}P{1cm}P{1cm}P{1cm}}
		\toprule	
		& 		 			& \multicolumn{5}{c}{\textbf{Train/Test Accuracy [\%]} 		}							       \\
		\multicolumn{2}{c}{\multirow{-2}{*}{Modelli}}
		& O 				& C 			   & E 				  & A 				 & N 			   \\ 
		\midrule
		\multirow{3}*{{Embedding 2}} 
		& \textbf{Modello 11} & \numprint{61}/\numprint{61} & \numprint{60}/\numprint{59} & \numprint{63}/\numprint{60} & \numprint{58}/\numprint{56} & \numprint{57}/\numprint{54} \\
		
		& \textbf{Modello 12} &\numprint{62}/\numprint{59} &\numprint{61}/\numprint{50} & \numprint{64}/\numprint{45} & \numprint{61}/\numprint{56} & \numprint{61}/\numprint{56} \\
		
		& \textbf{Modello 13} & \numprint{63}/\numprint{55} &\numprint{63}/\numprint{60} & \numprint{63}/\numprint{61} & \numprint{63}/\numprint{58} & \numprint{62}/\numprint{59} \\
		\midrule
		\multirow{2}*{{Embedding 3}} 
		& \textbf{Modello 14} & \numprint{61}/\numprint{61} & \numprint{61}/\numprint{57} & \numprint{62}/\numprint{52} & \numprint{60}/\numprint{57} & \numprint{60}/\numprint{56} \\
		
		& \textbf{Modello 15} &\numprint{63}/\numprint{60} & \numprint{62}/\numprint{48} & \numprint{64}/\numprint{60} & \numprint{62}/\numprint{49} & \numprint{62}/\numprint{48} \\
		\bottomrule	
	\end{tabular}
	\captionof{table}{Confronto delle accuracy sul test set delle diverse architetture}
	\label{tab:accuracymikolov}
\end{figure}

Il ``Modello 13'' sembrerebbe il migliore tra quelli proposti poiché presenta valori di accuracy più alti e la differenza di prestazioni tra train e test set è una tra le più basse, dimostrando minore propensione all'overfitting.  \\

Viene presentata un analisi più dettagliata su questo modello, in particolare mostrando il grafico dell'accuratezza, su train e test set, e le matrici di confusione calcolate per ogni tratto di personalità.

Come si può vedere chiaramente dalle matrici di confusione, le caratteristiche \emph{Conscientiousness} e \emph{Extraversion} sono quelle che vengono predette nel modo peggiore.

\begin{figure}[H]
	\centering
	\subfloat[][\emph{Visualizzazione matrice di confusione}\label{subfig:confO}]{
		\begin{minipage}[c][0.7\width]{0.45\textwidth}
			\centering
			\includegraphics[width=.6\textwidth]{images/binary/accO}
		\end{minipage}} 
	\hspace{10mm}
	\subfloat[][\emph{Visualizzazione accuratezza su train e test}\label{subfig:accO}]{
		\begin{minipage}[c][0.7\width]{0.45\textwidth}
			\centering
			\includegraphics[width=.8\textwidth]{images/binary/plotO}
		\end{minipage}}
	\caption{Analisi della caratteristica Openness del ``Modello 13''}
	\label{fig:binO}
\end{figure}

\begin{figure}[H]
	\centering
	\subfloat[][\emph{Visualizzazione matrice di confusione}\label{subfig:confC}]
	{\begin{minipage}[c][0.7\width]{0.45\textwidth}
			\centering
			\includegraphics[width=.6\textwidth]{images/binary/accC}
		\end{minipage}} 
	\hspace{10mm}
	\subfloat[][\emph{Visualizzazione accuratezza su train e test}\label{subfig:accC}]
	{\begin{minipage}[c][0.7\width]{0.45\textwidth}
			\centering
			\includegraphics[width=.8\textwidth]{images/binary/plotC}
		\end{minipage}}
	\caption{Analisi della caratteristica Conscientiousness del ``Modello 13''}
	\label{fig:binC}
\end{figure}

\begin{figure}[H]
	\centering
	\subfloat[][\emph{Visualizzazione matrice di confusione}\label{subfig:confE}]
	{\begin{minipage}[c][0.7\width]{0.45\textwidth}
			\centering
			\includegraphics[width=.6\textwidth]{images/binary/accE}
		\end{minipage}} 
	\hspace{10mm}
	\subfloat[][\emph{Visualizzazione accuratezza su train e test}\label{subfig:accE}]
	{\begin{minipage}[c][0.7\width]{0.45\textwidth}
			\centering
			\includegraphics[width=.8\textwidth]{images/binary/plotE}
		\end{minipage}}
	\caption{Analisi della caratteristica Extraversion del ``Modello 13''}
	\label{fig:binE}
\end{figure}

\begin{figure}[H]
	\centering
	\subfloat[][\emph{Visualizzazione matrice di confusione}\label{subfig:confA}]
	{\begin{minipage}[c][0.7\width]{0.45\textwidth}
			\centering
			\includegraphics[width=.6\textwidth]{images/binary/accA}
		\end{minipage}} 
	\hspace{10mm}
	\subfloat[][\emph{Visualizzazione accuratezza su train e test}\label{subfig:accA}]
	{\begin{minipage}[c][0.7\width]{0.45\textwidth}
			\centering
			\includegraphics[width=.8\textwidth]{images/binary/plotA}
		\end{minipage}}
	\caption{Analisi della caratteristica Agreeableness del ``Modello 13''}
	\label{fig:binA}
\end{figure}

\begin{figure}[H]
	\centering
	\subfloat[][\emph{Visualizzazione matrice di confusione}\label{subfig:confN}]
	{\begin{minipage}[c][0.7\width]{0.45\textwidth}
			\centering
			\includegraphics[width=.6\textwidth]{images/binary/accN}
		\end{minipage}} 
	\hspace{10mm}
	\subfloat[][\emph{Visualizzazione accuratezza su train e test}\label{subfig:accN}]
	{\begin{minipage}[c][0.7\width]{0.45\textwidth}
			\centering
			\includegraphics[width=.8\textwidth]{images/binary/plotN}
		\end{minipage}}
	\caption{Analisi della caratteristica Neuroticism del ``Modello 13''}
	\label{fig:binN}
\end{figure}


\chapter{Conclusioni}
\label{chap:conclusioni}

La natura di questo progetto di tesi è altamente sperimentale ed è volta a presentare analisi dettagliate sull'argomento, in quanto allo stato attuale non esistono importanti indagini che affrontino il problema dell'apprendimento dei tratti di personalità a partire da testo in linguaggio naturale.
\\

La domanda fondamentale che viene posta al fine di risolvere questo compito, è quale sia la metodologia adatta alla rappresentazione del testo.
Durante la sperimentazione, infatti, sono stati esaminati due principali approcci e ne è stata comparata l'efficacia.

Siamo partiti da una rappresentazione semplificata del testo ricorrendo al modello \emph{bag-of-words}. 
I risultati ottenuti sono sub-ottimali dal punto di vista della complessità. Inoltre la codifica utilizzata non fornisce alcuna informazione utile al sistema riguardo le relazioni che possono sussistere tra le parole di una frase.

In seguito è stata sfruttata una classe di algoritmi distribuzionali per insegnare alla rete il significato e le relazioni sussistenti tra le parole. Sfruttando la versione \emph{skip-gram} dell'algoritmo \texttt{word2vec} di Tomas Mikolov vengono rappresentate sotto forma di vettori le mappature tra parole e contesti nello spazio. 
Ricorrendo al secondo metodo i risultati ottenuti dimostrano come utilizzare un embedding sia la tecnica di estrazione di features più efficiente per filtrare le informazioni contenute in un testo.\\

È emerso che un'evoluzione di questo progetto potrebbe affacciarsi alla valutazione di rappresentazioni alternative del testo, annotazioni, part-of-speech e altre tecniche di NLP, per approfondire e migliorare ulteriormente l'indagine.\\

Dal punto di vista delle architetture implementate, abbiamo iniziato dall'implementazione di \emph{reti neurali fully-connected} come base per capire come modelli semplici di Deep Learning possano fornire informazioni sulle caratteristiche nascoste della personalità. 

Infine, ci addentriamo nelle \emph{reti neurali convoluzionali} molto più specializzate e efficiente delle precedenti nell'ambito del Text Mining.

Sviluppi futuri di questo lavoro potrebbero valutare una procedura di apprendimento alternativa, sfruttando altri modelli, quali le reti ricorrenti,  per ottenere dei risultati più efficaci.\\


La prima valutazione effettuata è definita come una regressione che tenta di prevedere l'esatto valore reale per ogni tratto di personalità.
Il problema presentato è estremamente complesso, e le performance ottenute sono ancora lontane da quelle desiderate.
Per questo motivo viene eseguita una seconda valutazione che trasforma il nostro compito in un problema di classificazione binaria multi-label.
I risultati raggiunti questa volta, evincono che ottimizzare la funzione di loss di un modello predittivo di regressione è molto più difficoltoso rispetto ad ottimizzare una funzione di costo stabile come la Softamax.
Questo è evidente poiché il modello di regressione tenta di produrre un esatto valore per ogni input, e i valori anomali predetti possono introdurre gradienti enormi.

 





\include{chap_quo}
\include{chap_qua}

\appendix
% INCLUSIONE APPENDICI -
\include{app_a}

%%%%%%%%%%%%%%%%%%%%%%%%%%%%%%%%%%%%%%%%%%%%%%%%%%%%%%%%%%%%%%%

% BIBLIOGRAFIA
\printbibliography
\addcontentsline{toc}{chapter}{\refname}
%\nocite{*}






\end{document}