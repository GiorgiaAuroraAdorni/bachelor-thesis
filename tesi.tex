%CLASSE DOCUMENTO - LINGUA E DIMENSIONE FONT
\documentclass[11pt]{toptesi}

%%%%%%%%%%%%%%%%%%%%%%%%%%%%%%%%%%%%%%%%%%%%%%%%%%%%%%%%%%%%%%%

% INCLUSIONE PACCHETTI
		
\usepackage{fullpage}	
\usepackage{graphicx}
\usepackage[pagestyles]{titlesec}
\usepackage[flushleft]{caption}
\usepackage{setspace}
\usepackage[utf8]{inputenc} %utf8 % lettere accentate da tastiera
\usepackage[italian]{babel} % lingua del documento
\usepackage[T1]{fontenc} % codifica dei font
\usepackage{blindtext}
\usepackage{graphicx,wrapfig}
\usepackage{booktabs}
\usepackage{lmodern}
\usepackage{varioref}
\usepackage{url}
\usepackage{array}
\usepackage{paralist}{\obeyspaces\global\let =\space}
\usepackage{verbatim} 
\usepackage{subfig}
\usepackage{tabularx}
\usepackage{amsmath}
\usepackage{amsfonts}
\usepackage{float}
\usepackage{amssymb}
\usepackage{multicol}
\usepackage{multirow}
\usepackage{listings}
\usepackage[pass]{geometry}
\usepackage[figuresright]{rotating}
\usepackage{algorithm}
\usepackage{algorithmic}
\usepackage{amsmath}
\usepackage[babel]{csquotes}
\usepackage{hyperref}
\usepackage[backend=bibtex]{biblatex}

%%%%%%%%%%%%%%%%%%%%%%%%%%%%%%%%%%%%%%%%%%%%%%%%%%%%%%%%%%%%%%%

% CONFIGURAZIONE LINK E RIFERIMENTI
\hypersetup{%
    pdfpagemode={UseOutlines},
    bookmarksopen,
    pdfstartview={FitH},
    colorlinks,
    linkcolor={black}, %COLORE DEI RIFERIMENTI AL TESTO
    citecolor={blue}, %COLORE DEI RIFERIMENTI ALLE CITAZIONI
    urlcolor={blue} %COLORI DEGLI URL
}

%%%%%%%%%%%%%%%%%%%%%%%%%%%%%%%%%%%%%%%%%%%%%%%%%%%%%%%%%%%%%%%

% CONFIGURAZIONE LISTATI/CODICE - CANCELLARE SE NON NECESSARIO
% PYTHON - BIANCO E NERO
\lstset{%
	captionpos=b,
	language=Python,
	basicstyle =\small\ttfamily,
	keywordstyle=\color{black}\bfseries,
	breaklines=true,
	breakatwhitespace=true,
	frame=lines,
	numbers=left,
	numberstyle=\footnotesize,
}

%%%%%%%%%%%%%%%%%%%%%%%%%%%%%%%%%%%%%%%%%%%%%%%%%%%%%%%%%%%%%%%

% FRENCHSPACING VA _SEMPRE_ ABILITATO PER DOCUMENTI IN ITALIANO
\frenchspacing

%%%%%%%%%%%%%%%%%%%%%%%%%%%%%%%%%%%%%%%%%%%%%%%%%%%%%%%%%%%%%%%

%DEFINIZIONE SEZIONI IN NUMERAZIONE ROMANA
%ELENCO DEI LISTATI/CODICI
\makeatletter
\newcommand\listofcodes{%
 \iffrontmatter\else\frontmattertrue\fi
 \if@openright\cleardoublepage\else\clearpage\fi
 % change the meaning of \chapter in a group
 \begingroup\def\chapter##1{\@schapter}
 \phantomsection % for the hyperlink
 \lstlistoflistings 
 \endgroup
} 
\makeatother

%%%%%%%%%%%%%%%%%%%%%%%%%%%%%%%%%%%%%%%%%%%%%%%%%%%%%%%%%%%%%%%

% INFORMAZIONI PDF - PERSONALIZZARE
\pdfinfo{%
  /Title    (Apprendimento della personalità basato sul linguaggio naturale)
  /Author   (Giorgia Adorni)
  /Subject  (Laura Informatica)
  /Keywords (Tesi)
}

%%%%%%%%%%%%%%%%%%%%%%%%%%%%%%%%%%%%%%%%%%%%%%%%%%%%%%%%%%%%%%%


% LISTA DEI CAPITOLI DA INCLUDERE - PERSONALIZZARE
\includeonly{%
frontespizio,%
abstract,%
chap_qui,%
chap_quo,%
chap_qua,%
app_a,%
}

%%%%%%%%%%%%%%%%%%%%%%%%%%%%%%%%%%%%%%%%%%%%%%%%%%%%%%%%%%%%%%%


% FILE DI BIBLIOGRAFIA
\bibliography{bibliography} 

% INIZIO DOCUMENTO
\begin{document}
	
% FRONTESPIZIO
\newpage
\pagestyle{empty} % no number
\noindent

\begin{figure}\doublespacing
	\mbox{
				\begin{minipage}{.20\textwidth}
					\includegraphics[height=3.3cm]{images/LogoBicocca.pdf}
				\end{minipage}%
				\quad\quad
				\begin{minipage}[c]{.90\textwidth}
					{Università degli Studi Milano Bicocca}\\
					{  \textbf{Scuola di Scienze}}\\
					{\textbf{Dipartimento di Informatica, Sistemistica e Comunicazione}}\\
					{  \textbf{Corso di Laurea in Informatica}}
				\end{minipage}%
	}

\end{figure}

\begin{center}
	\vspace{35mm}
\doublespacing\textbf{\huge RETI NEURALI PER~L’APPRENDIMENTO DEI TRATTI DELLA PERSONALITÀ DAL LINGUAGGIO NATURALE }\\
	
\end{center}

\vspace{30mm}
\onehalfspacing 

\begin{tabular}{ll}
	\textbf{Relatore: } & {Prof. Stella Fabio Antonio}\\
	\textbf{Co-relatore: } & {Dott. Marelli Marco}
\end{tabular}

\vspace{5mm}

\begin{flushright}\onehalfspacing 
	\textbf{Relazione della prova finale di:}\\
	{Giorgia Adorni}\\
	{Matricola 806787}\end{flushright}

\vspace{25mm}
\begin{center} \textbf{Anno Accademico 2017-2018 }\end{center}


%%%%%%%%%%%%%%%%%%%%%%%%%%%%%%%%%%%%%%%%%%%%%%%%%%%%%%%%%%%%%%%

%INTERLINEA - DEFAULT 1 - NON ESAGERATE, NON SUPERATE MAI 1.3 ;)
%\interlinea{1.2}

%%%%%%%%%%%%%%%%%%%%%%%%%%%%%%%%%%%%%%%%%%%%%%%%%%%%%%%%%%%%%%%

% DEDICA
% VSPACE - PROPORZIONE USATA PER CENTRATURA VERTICALE DEL TESTO
% FLUSHRIGHT - ALLINEAMENTO ORIZZONTALE A DESTRA
\vspace*{\stretch{1}}
\begin{flushright}
\noindent
\textit{A mio padre. }\\
\textit{Al mio fidanzato e collega. }
\end{flushright}
\vspace*{\stretch{6}}
\cleardoublepage


% CITAZIONE 
% VSPACE - PROPORZIONE USATA PER CENTRATURA VERTICALE DEL TESTO
% FLUSHRIGHT - ALLINEAMENTO ORIZZONTALE A DESTRA
\vspace*{\stretch{1}}
\begin{flushright}
\noindent
Citatemi dicendo che sono stato citato male.

\textit{Groucho Marx}
\end{flushright}
\vspace*{\stretch{6}}
\cleardoublepage

%%%%%%%%%%%%%%%%%%%%%%%%%%%%%%%%%%%%%%%%%%%%%%%%%%%%%%%%%%%%%%%

% RINGRAZIAMENTI - PERSONALIZZARE
\ringraziamenti
Grazie al mio relatore Fabio Stella e ai ragazzi del laboratorio MAD (Models and Algorithms for Data \& text mining).

%%%%%%%%%%%%%%%%%%%%%%%%%%%%%%%%%%%%%%%%%%%%%%%%%%%%%%%%%%%%%%%

% ABSTRACT - PERSONALIZZARE
\sommario
Apprendimento della personalità basato sul linguaggio naturale. \\
Reti neurali per la previsione dei tipi di personalità dagli stili di scrittura.
\\\\
La personalità è considerata come uno degli argomenti di ricerca più influenti in psicologia poiché predittiva di molti esiti consequenziali come la salute mentale e fisica, ed è in grado di spiegare il comportamento umano.
Grazie alla diffusione dei Social Network come mezzo di comunicazione, sta diventando sempre più importante sviluppare modelli che possano leggere automaticamente e con precisione l'essenza di individui basandosi esclusivamente sulla scrittura. 
\\\\
In particolare la convergenza tra scienze sociali (psicologiche) e informatiche hanno portato i ricercatori a sviluppare metodi automatizzati (approcci automatici) per estrarre e studiare le informazioni digitali "nascoste" nei dati testuali presenti in rete per prevedere i tratti della personalità.
\\\\
In questo studio, partendo da un dizionario di aggettivi che la letteratura psicologica definisce come marker dei cinque grandi tratti di personalità o Big Five, modello sul quale la maggior parte degli attuali studi automatici di rilevamento della personalità si sono concentrati, si vuole identificare un adeguato spazio semantico che permetta di definire la personalità dell'oggetto a cui un determinato testo si riferisce. 
\\\\
In questo lavoro, abbiamo esplorato vari metodi per affrontare il problema della predizione della personalità, partendo dall'implementazione di reti neurali feed-forward (fully-connected) come base per capire come i modelli semplici in deep-learning possano fornire informazioni sulle caratteristiche della personalità nascoste. 
\\\\
Infine sfruttando il concetto di word embedding, utilizziamo una classe di algoritmi distribuzionali, inventati nel 2013 da Tomas Mikolov, che consistono nell'utilizzo di un particolare tipo di rete neurale, detta ricorrente, che impara in modo non supervisionato i contesti delle parole.
In questo modo siamo in grado di trarre informazioni dalla semantica stessa del testo, e possiamo tradurre concetti in relazioni lineari, ottenendo una sorta di “geometria del significato”.
In quest'ultimo esperimento ipotizziamo che uno stile di scrittura individuale è in gran parte accoppiato con i tratti della sua personalità.

%%%%%%%%%%%%%%%%%%%%%%%%%%%%%%%%%%%%%%%%%%%%%%%%%%%%%%%%%%%%%%%

% INDICI 

% INDICE GENERALE
\tableofcontents

% INDICE DELLE FIGURE
\listoffigures

% INDICE DELLE TABELLE
\listoftables

% INDICE DEI CODICI
\listofcodes

%%%%%%%%%%%%%%%%%%%%%%%%%%%%%%%%%%%%%%%%%%%%%%%%%%%%%%%%%%%%%%%

% INCLUSIONE FILE CAPITOLI - TENERE COERENTE CON LISTA IN ALTO
\include{chap_qui}
\include{chap_quo}
\include{chap_qua}

\appendix
% INCLUSIONE APPENDICI -
\include{app_a}

%%%%%%%%%%%%%%%%%%%%%%%%%%%%%%%%%%%%%%%%%%%%%%%%%%%%%%%%%%%%%%%

% BIBLIOGRAFIA
\addcontentsline{toc}{chapter}{\refname}
\nocite{*}
\printbibliography

\end{document}